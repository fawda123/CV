\documentclass[letterpaper,12pt]{article} 
% \usepackage{times}
\usepackage[top=0.5in,bottom=0.5in,left=0.5in,right=0.5in]{geometry}
\usepackage{graphics}
\usepackage{booktabs}
\usepackage{fancyhdr}
\usepackage{xcolor}
\usepackage{enumitem}
\usepackage{url}
\usepackage{datetime}
\usepackage[colorlinks=true,urlcolor=blue,citecolor=blue,linkcolor=blue]{hyperref}

\newcommand{\sectitle}[1]{\vspace{\baselineskip} \centerline{\large{\textit{#1}}}} 
\newcommand{\subsectitle}[1]{\vspace{\baselineskip} \centerline{\normalsize{\textit{#1}}}} 
\newdateformat{mydate}{\monthname[\THEMONTH] \THEYEAR}
\urlstyle{same}

\setlist{topsep=0pt,parsep=6pt,partopsep=0pt,leftmargin=*} % list characteristics
\setlength{\parindent}{0pt}
\setlength{\parskip}{\baselineskip}%
\pagestyle{fancy}
\fancyhf{}
\fancyfoot[R]{\textcolor{gray}{Marcus W. Beck, CV page \thepage}}
\fancyfoot[L]{\textcolor{gray}{\mydate\today}}
\renewcommand{\headrulewidth}{0pt}
% \renewcommand{\rmdefault}{ptm}

\begin{document}

\raggedright

\LARGE
\centerline{{\bf Marcus W. Beck, Ph.D.}}
\normalsize
\textit{Tampa Bay Estuary Program \hfill Office: 727-893-2765 \\
263 13th Ave South \hfill Cell: 352-871-6795 \\
St. Petersburg, FL 33701 \hfill Email: \href{mailto:mbeck@tbep.org}{mbeck@tbep.org}}
\vspace{4pt}
\hrule
\vspace{2pt}
\hrule
\vspace{4pt}

\small

\sectitle{Education}

{\bf Ph.D., Conservation Biology, Fisheries and Aquatic Biology track} \hfill {\bf Jun. 2013} \\
{\bf Statistics minor} \\
Conservation Biology Graduate Program, University of Minnesota, Twin Cities, MN \\
Thesis title: Minnesota macrophytes: Linking aquatic plants, lake health, and human activities \\
Advisors: Drs. Bruce Vondracek, Lorin Hatch

{\bf M.Sc., Conservation Biology, Fisheries and Aquatic Biology track} \hfill {\bf Jun. 2009} \\
Conservation Biology Graduate Program, University of Minnesota, Twin Cities, MN \\
Thesis title: Development of an ecological assessment method for Minnesota lakes\\
Advisor: Dr. Lorin Hatch

{\bf B.Sc., Zoology} \textit{summa cum laude} \hfill {\bf Dec. 2006} \\
Department of Zoology, University of Florida, Gainesville, FL \\
Thesis title: Reproductive characteristics of the green treefrog \textit{Hyla cinerea}: effects of agricultural contaminants. \\
Advisor: Dr. Krista McCoy

{\bf A.A., Zoology} \hfill {\bf Apr. 2004} \\
Santa Fe Community College, Gainesville, FL

\sectitle{Professional Appointments}

{\bf Program Scientist} \hfill {\bf Nov. 2019 - Present}\\
Tampa Bay Estuary Program

{\bf Private Contractor} \hfill {\bf Apr. 2021 - Present}\\
EnviroDev, LLC

{\bf Scientist} \hfill {\bf Sep. 2017 - Sep. 2019}\\
Biology Department, Southern California Coastal Water Research Project

{\bf Post-Doctorate Researcher} \hfill {\bf Dec. 2015 - Sep. 2017}\\
Ecosystem Dynamics and Effects Branch, Gulf Ecology Division, United States Environmental Protection Agency

{\bf Post-Doctorate Research Fellow} \hfill {\bf Jul. 2013 - Nov. 2015}\\
Oak Ridge Institute for Science and Education, Ecosystem Dynamics and Effects Branch, Gulf Ecology Division, United States Environmental Protection Agency

{\bf Research Fellow} \hfill {\bf Sep. 2012 - Jun. 2013} \\
Water Resources Center, Conservation Biology Graduate Program, Department of Fisheries, Wildlife, and Conservation Biology, University of Minnesota

{\bf Research Assistant}, University of Minnesota \hfill {\bf Sep. 2007 - Sep. 2012} \\
Conservation Biology Graduate Program, Department of Fisheries, Wildlife, and Conservation Biology, University of Minnesota

{\bf Intern} \hfill {\bf Jun. 2009 - Aug. 2009} \\
Minnesota Department of Natural Resources

{\bf Creel Clerk} \hfill {\bf Jan. 2007 - May 2007} \\
Florida Fish and Wildlife Conservation Commission

{\bf Undergraduate Researcher} \hfill {\bf Jan. 2006 - Dec. 2006} \\
Department of Zoology, University of Florida

{\bf Biological Technician} \hfill {\bf summer 2006} \\
Florida Integrated Science Center, United States Geological Survey

\sectitle{Teaching Experience}

{\bf Workshop Intructor} \hfill {\bf May 2023}\\
Open Science: An Introduction for Fisheries Professionals, Florida chapter of the American Fisheries Society, Annual Meeting, Saint Augustine, Florida ({\footnotesize\href{https://tbep-tech.github.io/flafs-os-workshop/}{link}})

{\bf Workshop Co-Intructor} \hfill {\bf Mar. 2023}\\
MassWateR R training, Essex, MA ({\footnotesize\href{https://massbays-tech.github.io/intro-to-r/}{link}})

{\bf Workshop Co-Intructor} \hfill {\bf Feb. 2023}\\
Open Science: Hands On, In Person Training for Natural Resource Manageers, SWFWMD Tampa Service Office ({\footnotesize\href{https://tbep-tech.github.io/tbep-os-workshop/}{link}})

{\bf Remote Workshop Co-Intructor} \hfill {\bf Nov. 2021}\\
Open Science: Core Concepts for Impactful Research and Resource Management, Coastal and Estuarine Research Federation Virtual Meeting ({\footnotesize\href{https://tbep-tech.github.io/cerf-os-workshop/}{link}})

{\bf Remote Workshop Instructor} \hfill {\bf Apr. 2021}\\
NERRS Ecosystem Metabolism R Training, Remote Webinar ({\footnotesize\href{https://tbep-tech.github.io/ecometab-r-training/}{link}})

{\bf Remote Workshop Instructor} \hfill {\bf Feb. 2021}\\
Peconic Estuary Program R Training, Remote Webinar ({\footnotesize\href{https://tbep-tech.github.io/pep-r-training/}{link}})

{\bf Remote Workshop Instructor} \hfill {\bf Jun. 2020}\\
Tampa Bay Estuary Program R Training, Remote Webinar ({\footnotesize\href{https://tbep-tech.github.io/tbep-r-training/}{link}})

{\bf Workshop Co-Instructor} \hfill {\bf Sep. 2019}\\
Tampa Bay Estuary Program Open Science Workshop, St. Petersburg College STEM Center ({\footnotesize\href{https://tbep-tech.github.io/}{link}})

{\bf Workshop Co-Instructor} \hfill {\bf Oct. 2018}\\
Special Cal SFS workshop: Visualizing and mapping bioassessment data in R, California Aquatic Bioassessment Workgroup ({\footnotesize\href{https://sccwrp.github.io/CABW2018_R_training}{link}})

{\bf Workshop Co-Instructor} \hfill {\bf May 2018}\\
Fundamentals of spatial data access and analysis in R, Annual meeting of the Society of Freshwater Science ({\footnotesize\href{https://ryan-hill.github.io/sfs-r-gis-2018/}{link}})

{\bf Workshop Co-Instructor} \hfill {\bf Nov. 2017}\\
Introduction to R for analysis of coastal and estuarine data, Biennial meeting of the Coastal and Estuarine Research Federation ({\footnotesize\href{https://usepa.github.io/cerf_r/}{link}})

{\bf Workshop Co-Instructor} \hfill {\bf Nov. 2016}\\
R, SWMPr, SWMPrats for non-beginners, System Wide Monitoring Program, National Estuarine Research Reserve System, NOAA. ({\footnotesize\href{http://swmprats.net/2016-workshop}{link}})

{\bf Workshop Co-Instructor} \hfill {\bf Oct. 2015}\\
An Introduction to SWMPr, System Wide Monitoring Program, National Estuarine Research Reserve System, NOAA. ({\footnotesize\href{http://swmprats.net/workshop-2015}{link}})

{\bf Workshop Co-Instructor} \hfill {\bf Nov. 2014}\\
NERRS/SWMP Data Analysis (Time-series) Workshop, System Wide Monitoring Program, National Estuarine Research Reserve System, NOAA.

{\bf Workshop Co-Instructor} \hfill {\bf May 2013} \\
Using Available Software for Basic Analysis, Restoring Minnesota, University of Minnesota.

{\bf Primary Instructor}, University of Minnesota \hfill {\bf Jan. 2012 - May 2012} \\
FW 5604 (Fisheries Ecology and Management), Department of Fisheries, Wildlife, and Conservation Biology, University of Minnesota

{\bf Undergraduate Advisor}, University of Minnesota \hfill {\bf Mar. 2009 - May 2011} \\
Justine Koch (Gustavus Adolphus College), Kristan Maccaroni (University of Minnesota)

{\bf Teaching Assistant}, University of Minnesota \hfill {\bf Sep. 2008 - Dec. 2008} \\
ESPM 4061/5061 (Water Quality and Natural Resources), Department of Environmental Sciences, Policy, and Management, University of Minnesota

\sectitle{Service}

{\bf Reviewer}\hfill \\
\textit{Acta Oecologica}, 2015 $\bullet$ \textit{Aquatic Botany}, 2011, 2014 $\bullet$ \textit{Aquatic Ecology}, 2018 $\bullet$ \textit{Aquatic Sciences}, 2015, 2018, 2020 $\bullet$ \textit{Chemosphere}, 2019 $\bullet$ \textit{Coastal Management}, 2020 $\bullet$ \textit{Development in Practice}, 2012, 2013 $\bullet$ \textit{Ecological Indicators}, 2010, 2019 $\bullet$ \textit{Ecological Modelling}, 2016, 2020(2) $\bullet$ \textit{Ecological Research}, 2012 $\bullet$ Elsevier book chapter, 2014 $\bullet$ \textit{Engineering Computations}, 2016 $\bullet$ \textit{Environmental Management}, 2010 $\bullet$ \textit{Environmental Modelling \& Software}, 2015 $\bullet$ \textit{Environmental Science \& Technology}, 2019, 2021 $\bullet$ \textit{Environments}, 2019 $\bullet$ \textit{Estuaries and Coasts}, 2014, 2015, 2018, 2022, 2023 $\bullet$ \textit{Estuarine, Coastal and Shelf Science}, 2016(2), 2017 $\bullet$ \textit{Expert Systems with Applications}, 2019 $\bullet$ \textit{Freshwater Science}, 2019, 2021 $\bullet$ \textit{Frontiers in Ecology and the Environment}, 2019 $\bullet$ \textit{Frontiers in Ecology and Evolution}, 2021 $\bullet$ \textit{Frontiers in Environmental Science}, 2022(3) $\bullet$ \textit{Frontiers in Marine Science}, 2023 (2) $\bullet$ \textit{Frontiers in Plant Science}, 2017 $\bullet$ \textit{Geocarto International}, 2017 $\bullet$ \textit{Heliyon}, 2022 $\bullet$ \textit{Hydrobiologia}, 2012 $\bullet$ \textit{International Journal of Tropical Biology and Conservation}, 2013 $\bullet$ \textit{International Journal of River Basin Management}, 2013 $\bullet$ \textit{ISPRS Journal of Photogrammetry and Remote Sensing}, 2017 $\bullet$ \textit{Journal of Applied Phycology}, 2015 $\bullet$ \textit{Journal of Environmental Management}, 2020, 2022 $\bullet$ \textit{Journal of the American Water Resources Association}, 2020 $\bullet$ \textit{Journal of the North American Benthological Society}, 2011 $\bullet$ \textit{Journal of Statistical Software}, 2020 $\bullet$ \textit{Lake and Reservoir Management}, 2013, 2014, 2015, 2017(2) $\bullet$ \textit{Limnology \& Oceanography}, 2013, 2023 $\bullet$ \textit{Marine Chemistry}, 2020 $\bullet$ \textit{Marine Ecology}, 2014 $\bullet$ \textit{Marine Pollution Bulletin}, 2014, 2021(2), 2023 $\bullet$ Minnesota Department of Natural Resources, 2011 $\bullet$ \textit{New Zealand Journal of Marine and Freshwater Research}, 2019 $\bullet$ \textit{North American Journal of Fisheries Management}, 2010, 2011, 2012, 2016 $\bullet$ \textit{PeerJ}, 2018 $\bullet$ \textit{PLOS Computational Biology}, 2021 $\bullet$ \textit{Polar Biology}, 2011 $\bullet$ \textit{The R Journal}, 2016(3) $\bullet$ \textit{Remote Sensing}, 2014, 2015, 2016(2), 2017 $\bullet$ \textit{rOpenSci}, 2016, 2021 $\bullet$ \textit{Science of the Total Environment}, 2011 $\bullet$ \textit{Scientific Reports}, 2018 $\bullet$ \textit{Thalassas}, 2018 $\bullet$ USEPA Office of Research and Development, 2014, 2015, 2021 $\bullet$ \textit{Water Research}, 2015, 2016, 2022

{\bf Editorial Board} \hfill {\bf Jan. 2022 - Present} \\
Coastal and Estuarine Science News (\href{https://www.cerf.science/cesn}{link})

{\bf Grant Reviewer} \hfill {\bf Oct. 2022} \\
Florida RESTORE Act Centers of Excellence Program, RFP IV

{\bf Grant Reviewer} \hfill {\bf Jul. 2022} \\
Florida Fish and Wildlife Commission 2022 - 2024 Harmful Algal Bloom Grants Program

{\bf Graduate Student Representative} \hfill {\bf Jan. 2013 - May 2013} \\
Department of Fisheries, Wildlife, and Conservation Biology, University of Minnesota

{\bf Student Activities Committee Co-Chair} \hfill {\bf Jun. 2010 - Aug. 2012} \\
2012 annual meeting of the American Fisheries Society

{\bf Fundraising Coordinator} \hfill {\bf spring 2011, spring 2012} \\
Conservation Biology Graduate Program, University of Minnesota

{\bf Seminar Coordinator} \hfill {\bf 2009, fall 2011, spring 2012} \\
Conservation Biology Graduate Program, Department of Fisheries, Wildlife, and Conservation Biology, University of Minnesota

{\bf Student Representative} \hfill {\bf Aug. 2009 - Aug. 2012} \\
Minnesota Chapter of the American Fisheries Society

{\bf Student Club Officer} \hfill {\bf Sep. 2010 - May 2012} \\
Fisheries and Wildlife Club, Department of Fisheries, Wildlife, and Conservation Biology, University of Minnesota 

{\bf Lab Volunteer} \hfill {\bf summer 2005} \\
Florida Museum of Natural, University of Florida

\sectitle{Peer-reviewed Publications}

\subsectitle{In Review}

\begin{enumerate}

\item Gilliam, F.S., Murrell, M.C., \textbf{Beck, M.W.}. \textbf{In review}. Environmental threats to the State of Florida - climate change beyond: Volume II. \textit{Frontiers in Ecology and Evolution}.

\item Meyers, S., \textbf{Beck, M.W.}, Sherwood, E.T., Luther, M. \textbf{In review}. Developing a genetic algorithm for selecting infrastructure modifications that optimize hydrodynamic flushing in Tampa Bay. \textit{Environmental Modelling and Software}. 

\item Bednar\v{s}ek, N., Pelletier, G., \textbf{Beck, M.W.}, Feely, R., Siegrist, Z., Kiefer, D., Davis, J., Peadoby, B. \textbf{In review}. Predictable patterns within the kelp forest can indirectly create temporary spatial refugia for ocean acidification. \textit{Environmental Science \& Technology}.

\item Smith, J., Eggleston, E., Howard, M.D.A., Ryan, S., Gichuki, J., Kennedy, K., Tyler, A., \textbf{Beck, M.W.}, Huie, S., Caron, D.A. \textbf{In review}. Historic and recent trends of cyanobacterial harmful algal blooms and environmental conditions in Clear Lake, California: a 70-year perspective. \textit{Elementa: Science of the Anthropocene}.

\end{enumerate}

\subsectitle{In Press or Published}

\begin{enumerate}

\item Meyers, S., Luther, M., Landry, S., \textbf{Beck, M.W.} \textbf{In press} A simple machine learning approach to modeling sanitary sewer overflows in southern Pinellas County, FL. \textit{Florida Scientist}. 

\item \textbf{Beck, M.W.}, Burke, M.C., Raulerson, G.E., Scolaro, S., Sherwood, E.T., Whalen, J. \textbf{In press}. Coordinated monitoring of the Piney Point wastewater discharge into Tampa Bay: Data synthesis and reporting. \textit{Florida Scientist}.

\item Wessel, M.R., \textbf{Beck, M.W.}, Sherwood, E.T., Peebles, E., Hall, E. \textbf{In press}. Establishing a community of practice for tidal creek research using conceptual models and open science. \textit{Florida Scientist}. 

\item Scolaro, S., \textbf{Beck, M.W.}, Burke, M.C., Raulerson, G.E., Sherwood, E.T. \textbf{In press}. Piney Point, seagrass, and macroalgae: Impact assessment and a case for enhanced macroalgae monitoring. \textit{Florida Scientist}.

\item Morrison, E.S., Phlips, E.J., Badylak, S., Chappel, A.R., Altieri, A.H., Osborne, T.Z., Tomasko, D.A., \textbf{Beck, M.W.}, Sherwood, E.T. 2023. The response of Tampa Bay to a legacy mining nutrient release in the year following the event. \textit{Frontiers in Ecology and Evolution}. 11:1144778. ({\footnotesize\href{https://doi.org/10.3389/fevo.2023.1144778}{10.3389/fevo.2023.1144778}})

\item \textbf{Beck, M.W.}, Robison, D.E., Raulerson, G.E., Burke, M.C., Saarinen, J.A., Sciarrino, C.M., Sherwood, E.T., Tomasko, D.A. 2023. Addressing climate change and development pressures in an urban estuary through habitat restoration planning. \textit{Frontiers in Ecology and Evolution}. 11:1070266. ({\footnotesize\href{https://doi.org/10.3389/fevo.2023.1070266}{10.3389/fevo.2023.1070266}})

\item Mazor, R.D., Sutula, M., Theroux, S., \textbf{Beck, M.W.}, Ode, P.R. 2022. Eutrophication thresholds associated with protection of biological integrity in California wadeable streams. \textit{Ecological Indicators}. 142:109180. ({\footnotesize\href{https://doi.org/10.1016/j.ecolind.2022.109180}{10.1016/j.ecolind.2022.109180}})

\item Bednar\v{s}ek, N., \textbf{Beck, M.W.}, Applebaum, S.L., Feely, R.A., Pelletier, G., Butler, R., Byrne, M., Peabody, B., Davis, J., \v{S}trus, J. 2022. Natural analogues in p{H} variability and predictability across the {C}oastal {P}acific estuaries: extrapolation of the increased oyster dissolution under increased p{H} amplitude and low predictability related to ocean acidification. \textit{Environmental Science \& Technology}. 56(12):9015-9028. ({\footnotesize\href{https://doi.org/10.1021/acs.est.2c00010}{10.1021/acs.est.2c00010}})

\item Sun, H., Le, C., Long, S., \textbf{Beck, M.W.} 2022. Linking phytoplankton variability to atmospheric blocking in an eastern boundary upwelling system. \textit{Journal of Geophysical Research - Oceans}. 127(6):e2021JC017348. ({\footnotesize\href{https://doi.org/10.1029/2021JC017348}{10.1029/2021JC017348}})

\item \textbf{Beck, M.W.}, Altieri, A., Angelini, C., Burke, M.C., Chen, J., Chin, D.W., Gardiner, J., Hu, C., Hubbard, K.A., Liu, Y., Lopez, C., Medina, M., Morrison, E., Phlips, E.J., Raulerson, G.E., Scolaro, S., Sherwood, E.T., Tomasko, D., Weisberg, R., Whalen, J. 2022. Initial estuarine response to the nutrient-rich Piney Point release into Tampa Bay, Florida. \textit{Marine Pollution Bulletin}. 178:113598. ({\footnotesize\href{https://doi.org/10.1016/j.marpolbul.2022.113598}{10.1016/j.marpolbul.2022.113598}})

\item Wessel, M.R., Leverone, J.R., \textbf{Beck, M.W.}, Sherwood, E.T., Hecker, J., West, S., Janicki, A. 2022. A nutrient-based framework for tidal creeks in Southwest Florida. \textit{Estuaries and Coasts}. 45:17-37. ({\footnotesize\href{https://doi.org/10.1007/s12237-021-00974-7}{10.1007/s12237-021-00974-7}})

\item \textbf{Beck, M.W.}, de Valpine, P., Murphy, R., Wren, I., Chelsky, A., Foley, M., Senn, D.B. 2022. Multi-scale trend analysis of water quality using error propagation of generalized additive models. \textit{Science of the Total Environment}. 802:149927. ({\footnotesize\href{https://doi.org/10.1016/j.scitotenv.2021.149927}{10.1016/j.scitotenv.2021.149927}})

\item Skripnikov, A., Wagner, N., Shafer, J., \textbf{Beck, M.W.}, Sherwood, E.T., Burke, M. 2021. Using localized Twitter activity for red tide impact assessment. \textit{Harmful Algae}. 110:102118. ({\footnotesize\href{https://doi.org/10.1016/j.hal.2021.102118}{10.1016/j.hal.2021.102118}})

\item \textbf{Beck, M.W.}, Schrandt, M.N., Wessel, M.R., Sherwood, E.T., Raulerson, G.E., Prasad, A.A.B., Best, B.D. 2021. tbeptools: An R package for synthesizing estuarine data for environmental research. \textit{Journal of Open Source Software}. 6(65):3485. ({\footnotesize\href{https://doi.org/10.21105/joss.03485}{10.21105/joss.03485}})

\item Otim, O., \textbf{Beck, M.W.} 2021. Multivariate analysis of sediment toxicity in an ocean ecosystem: A Southern California Bight case study. \textit{Environmental Science \& Technology}. 55(17):12116-12125. ({\footnotesize\href{https://doi.org/10.1021/acs.est.1c03032}{10.1021/acs.est.1c03032}})

\item Rogers, J.B., Stein, E.D., \textbf{Beck, M.W.}, Flint, K., Kinoshita, A.M., Ambrose, R.F. 2021. Modeling future changes to the hydrological and thermal regime of unaltered streams using projected changes in climate. \textit{Ecohydrology}. 14(5):e2299. ({\footnotesize\href{https://doi.org/10.1002/eco.2299}{10.1002/eco.2299}})

\item Bednar\v{s}ek, N., Newton, J., \textbf{Beck, M.W.}, Alin, S.R., Feely, R.A., Christman, N., Klinger, T. 2021. Severe biological effects under present-day estuarine acidification in the highly variable {S}alish {S}ea. \textit{Science of the Total Environment}. 765:142689. ({\footnotesize\href{https://doi.org/10.1016/j.scitotenv.2020.142689}{10.1016/j.scitotenv.2020.142689}})

\item Schrandt, M.N., MacDonald, T.C., Sherwood, E.T., \textbf{Beck, M.W.} 2021. A multimetric index for monitoring, managing and communicating ecosystem health status in an urbanized Gulf of Mexico estuary. \textit{Ecological Indicators}. 123:107310. ({\footnotesize\href{https://doi.org/10.1016/j.ecolind.2020.107310}{10.1016/j.ecolind.2020.107310}})

\item Meyers, S., Landry, S., \textbf{Beck, M.W.}, Luther, M. 2021. Using logistic regression to model the risk of sewer overflows triggered by compound flooding with application to sea level rise. \textit{Urban Climate}. 35:100752. ({\footnotesize\href{https://doi.org/10.1016/j.uclim.2020.100752}{10.1016/j.uclim.2020.100752}})

\item Rogers, J.B., Stein, E.D., \textbf{Beck, M.W.}, Ambrose, R.F. 2020. The impact of climate change induced alterations of streamflow and stream temperature on the distribution of riparian species. \textit{PLoS ONE}. 15(11):e0242682. ({\footnotesize\href{https://doi.org/10.1371/journal.pone.0242682}{10.1371/journal.pone.0242682}})

\item Theroux, S., Mazor, R.D., \textbf{Beck, M.W.}, Ode, P., Stein, E., Sutula, M., 2020. Predictive biological indices for algae populations in diverse landscapes. \textit{Ecological Indicators}. 119:106421. ({\footnotesize\href{https://doi.org/10.1016/j.ecolind.2020.106421}{10.1016/j.ecolind.2020.106421}})

\item Larkin, D.J., \textbf{Beck, M.W.}, Bajer, P.G. 2020. An invasive fish promotes invasive plants in Minnesota lakes. \textit{Freshwater Biology}. 65(9):1608-1621. ({\footnotesize\href{https://doi.org/10.1111/fwb.13526}{10.1111/fwb.13526}})

\item \textbf{Beck, M.W.}, O'Hara, C., Stewart Lowndes, J., Mazor, R.D., Theroux, S.T., Gillett, D.J., Lane, B., Gearheart, G. 2020. The importance of open science for biological assessment. \textit{PeerJ}:e9539. ({\footnotesize\href{https://doi.org/10.7717/peerj.9539}{10.7717/peerj.9539}})

\item Bednar\v{s}ek, N., Feely, R.A., \textbf{Beck, M.W.}, Alin, S.R., Siedlecki, S.A., Calosi, P., Norton, E.L., Saenger, C., \v{S}trus, J., Greeley, D., Nezlin, N.P., Spicer, J.I. 2020. Carapace dissolution, growth decline and mechanorecepter damages in {D}ungeness crab related to severity of ocean acidification gradients. \textit{Science of the Total Environment}. 716:136610 ({\footnotesize\href{https://doi.org/10.1016/j.scitotenv.2020.136610}{10.1016/j.scitotenv.2020.136610}})

\item \textbf{Beck, M.W.}, Mazor, R.D., Johnson, S., Wisenbaker, K., Westfall, J., Ode, P.R., Hill, R., Loflen, C., Sutula, M., Stein, E.D. 2019. Prioritizing management goals for stream biological integrity within the developed landscape context. \textit{Freshwater Science}. 38(4):883-898. ({\footnotesize\href{https://doi.org/10.1086/705996}{10.1086/705996}})

\item \textbf{Beck, M.W.}, Sherwood, E.T., Henkel, J.R., Dorans, K., Ireland, K., Varela, P. 2019. Assessment of the cumulative effects of restoration activities on water quality in Tampa Bay, Florida. \textit{Estuaries and Coasts}. 42(7):1774-1791. ({\footnotesize\href{https://doi.org/10.1007/s12237-019-00619-w}{10.1007/s12237-019-00619-w}})

\item \textbf{Beck, M.W.}, Mazor, R.D., Theroux, S.T., Schiff, K.C. 2019. The Stream Quality Index: A multi-indicator tool for enhancing environmental management communication. \textit{Environmental and Sustainability Indicators}. 1(2):100004. ({\footnotesize\href{https://doi.org/10.1016/j.indic.2019.100004}{10.1016/j.indic.2019.100004}})

\item Le, C., Wu, S., Hu, C., \textbf{Beck, M.W.}, Yang, X. 2019. Phytoplankton decline in the eastern {N}orth {P}acific transition zone associated with atmospheric blocking. \textit{Global Change Biology}. 25(10):3485-3493. ({\footnotesize\href{https://doi.org/10.1111/gcb.14737}{10.1111/gcb.14737}})

\item Bednar\v{s}ek, N., Feely, R.A., \textbf{Beck, M.W.}, Glippa, O., Kanerva, M., Engstr\"{o}m-\"{O}st, J. 2018. El Ni\~{n}o-related thermal stress coupled with ocean acidification negatively impacts cellular to population-level responses in pteropods along the {C}alifornia {C}urrent {S}ystem with implications for increased bioenergetic costs. \textit{Frontiers in Marine Science}. 5(486):1-17. ({\footnotesize\href{https://doi.org/10.3389/fmars.2018.00486}{10.3389/fmars.2018.00486}})

\item Bokde, N., Mart\'{i}nez \'{A}lvarez, F., \textbf{Beck, M.W.}, Kulat, K. 2018. A novel imputation methodology for time series based on pattern sequence forecasting. \textit{Pattern Recognition Letters}. 116:88-96. ({\footnotesize\href{https://doi.org/10.1016/j.patrec.2018.09.020}{10.1016/j.patrec.2018.09.020}})

\item \textbf{Beck, M.W.} 2018. NeuralNetTools: Visualization and analysis tools for neural networks. \textit{Journal of Statistical Software}. 85(11):1-20. ({\footnotesize\href{http://dx.doi.org/10.18637/jss.v085.i11}{10.18637/jss.v085.i11}})

\item \textbf{Beck, M.W.}, Jabusch, T.W., Trowbridge, P.R., Senn, D.B. 2018. Four decades of water quality change in the upper San Francisco Estuary. \textit{Estuarine, Coastal and Shelf Science}. 212:11-22. ({\footnotesize\href{https://doi.org/10.1016/j.ecss.2018.06.021}{10.1016/j.ecss.2018.06.021}})

\item Zhang, Z., \textbf{Beck, M.W.}, Winkler, D. A., Huang, B., Sibanda, W., Goyal, H. 2018. Opening the black box of neural networks: methods for interpreting neural network models in clinical applications. \textit{Annals of translational medicine}. 6(11):1-11. ({\footnotesize\href{http://dx.doi.org/10.21037/atm.2018.05.32}{10.21037/atm.2018.05.32}})

\item \textbf{Beck, M.W.}, Bokde, N., Asencio-Cort\'{e}s, G., Kulat, K.D. 2018. R package imputeTestbench to compare imputation methods for univariate time series. \textit{The R Journal}. 1-16.

\item Bajer, P.G., \textbf{Beck, M.W.}, Hundt, P. 2018. Effect of non-native versus native invaders on macrophyte richness: are carp and bullheads ecological proxies? \textit{Hydrobiologia}. 817(1):379-391. ({\footnotesize\href{https://link.springer.com/article/10.1007/s10750-018-3592-1}{10.1007/s10750-018-3592-1}})

\item Murrell, M.C., Caffrey, J.M., Marcovich, D.T., \textbf{Beck, M.W.}, Jarvis, B.M., Hagy, J.D. 2018. Seasonal oxygen dynamics in a warm temperate estuary: Effects of hydrologic variability on measurements of primary production, respiration, and net metabolism. \textit{Estuaries and Coasts}. 41(3):690-707. ({\footnotesize\href{https://link.springer.com/article/10.1007/s12237-017-0328-9}{10.1007/s12237-017-0328-9}})

\item \textbf{Beck, M.W.}, Cressman, K., Griffin, C., Caffrey, J. 2018. Water quality trends following anomalous phosphorus inputs to Grand Bay, Mississippi, USA. \textit{Gulf and Caribbean Research}. 29(1):1-14. ({\footnotesize\href{https://aquila.usm.edu/gcr/vol29/iss1/2/}{gcr.2901.02}})

\item \textbf{Beck, M.W.}, Hagy, J.D., Le, C. 2018. Quantifying seagrass light requirements using an algorithm to spatially resolve depth of colonization. \textit{Estuaries and Coasts}. 41(2):592-610. ({\footnotesize\href{http://dx.doi.org/10.1007/s12237-017-0287-1}{10.1007/s12237-017-0287-1}})

\item Sutula, M., Kudela, R., Hagy, J.D., Harding, L.W., Senn, D., Cloern, J.E., Bricker, S., Berg, G.M., \textbf{Beck, M.W.} 2017. Novel analyses of long-term data provide a scientific basis for chlorophyll-a thresholds in San Francisco Bay. \textit{Estuarine, Coastal and Shelf Science}. 197:107-118. ({\footnotesize\href{https://doi.org/10.1016/j.ecss.2017.07.009}{10.1016/j.ecss.2017.07.009}})

\item \textbf{Beck, M.W.}, Lehrter, J.C., Lowe, L.L., Jarvis, B.M. 2017. Parameter sensitivity and identifiability for a biogeochemical model of hypoxia in the northern Gulf of Mexico. \textit{Ecological Modelling}. 363:17-30. ({\footnotesize\href{http://dx.doi.org/10.1016/j.ecolmodel.2017.08.020}{10.1016/j.ecolmodel.2017.08.020}})

\item \textbf{Beck, M.W.}, Alahuhta, J. 2017. Ecological determinants of \textit{Potamogeton} taxa in glacial lakes: assemblage composition, species richness, and species-level approach. \textit{Aquatic Sciences}. 79(3):427-441. ({\footnotesize\href{https://doi.org/10.1007/s00027-016-0508-x}{10.1007/s00027-016-0508-x}})

\item Le, C., Lehrter, J.C., Hu, C., MacIntyre, H., \textbf{Beck, M.W.} 2017. Satellite observation of particulate organic carbon dynamics in two river-dominated estuaries. \textit{Journal of Geophysical Research: Oceans}. 122(1):555-569. ({\footnotesize\href{http://dx.doi.org/10.1002/2016JC012275}{10.1002/2016JC012275}})

\item \textbf{Beck, M.W.}, Murphy, R. 2017. Numerical and qualitative contrasts of two statistical models for water quality change in tidal waters. \textit{Journal of the American Water Resources Association}. 53(1):197-219. ({\footnotesize\href{http://dx.doi.org/10.1111/1752-1688.12489}{10.1111/1752-1688.12489}})

\item Bajer, P.G., \textbf{Beck, M.W.}, Cross, T.K., Koch, J., Bartodziek, B., Sorensen, P.W. 2016. Biological invasion by a benthivorous fish reduced the cover and species richness of aquatic plants in most lakes of a large North American ecoregion. \textit{Global Change Biology}. 22(12):3937-3947. ({\footnotesize\href{http://dx.doi.org/10.1111/gcb.13377}{10.1111/gcb.13377}})

\item \textbf{Beck, M.W.} 2016. SWMPr: An R package for retrieving, organizing, and analyzing environmental data for estuaries. \textit{The R Journal}. 8(1):219-232.

\item Le, C., Lehrter, J.C., Schaeffer, B.A., Hu, C., Murrell, M.C., Hagy, J.D., Greene, R.M., \textbf{Beck, M.W.} 2016. Bio-optical water quality dynamics observed from MERIS in Pensacola Bay, Florida. \textit{Estuarine, Coastal and Shelf Science}. 173:26-38. ({\footnotesize\href{https://doi.org/10.1016/j.ecss.2016.02.003}{10.1016/j.ecss.2016.02.003}})

\item \textbf{Beck, M.W.}, Hagy, J.D., Murrell, M.C. 2015. Improving estimates of ecosystem metabolism by reducing effects of tidal advection on dissolved oxygen time series. \textit{Limnology \& Oceanography: Methods}. 13(12):731-745. ({\footnotesize\href{http://dx.doi.org/10.1002/lom3.10062}{10.1002/lom3.10062}})

\item \textbf{Beck, M.W.}, Hagy, J.D. 2015. Adaptation of a weighted regression approach to evaluate water quality trends in an estuary. \textit{Environmental Modeling and Assessment}. 20(6):637-655. ({\footnotesize\href{http://dx.doi.org/10.1007/s10666-015-9452-8}{10.1007/s10666-015-9452-8}})

\item \textbf{Beck, M.W.}, Tomcko, C.M., Valley, R.D., Staples, D.F. 2014. Analysis of macrophyte indicator variation as a function of sampling, temporal, and stressor effects. \textit{Ecological Indicators}. 46:323-335. ({\footnotesize\href{https://doi.org/10.1016/j.ecolind.2014.07.002}{10.1016/j.ecolind.2014.07.002}})

\item \textbf{Beck, M.W.}, Wilson, B.N., Vondracek, B., Hatch, L.K. 2014. Application of neural networks to quantify utility of indices of biotic integrity for biological monitoring. \textit{Ecological Indicators}. 45:195-208. ({\footnotesize\href{https://doi.org/10.1016/j.ecolind.2014.04.002}{10.1016/j.ecolind.2014.04.002}})

\item Vondracek, B., Koch, J.D., and \textbf{Beck, M.W.} 2014. A comparison of survey methods to evaluate macrophyte index of biotic integrity performance in Minnesota lakes. \textit{Ecological Indicators}. 36:178-185. ({\footnotesize\href{https://doi.org/10.1016/j.ecolind.2013.07.002}{10.1016/j.ecolind.2013.07.002}})

\item \textbf{Beck, M.W.}, Vondracek, B. Hatch, L.K. 2013. Between- and within-lake responses of macrophyte richness metrics to shoreline development. \textit{Lake and Reservoir Management}. 29(3):179-193. ({\footnotesize\href{http://dx.doi.org/10.1080/10402381.2013.828806}{10.1080/10402381.2013.828806}})

\item \textbf{Beck, M.W.}, Vondracek, B., Hatch, L.K., and Vinje, J. 2013. Semi-automated analysis of high-resolution aerial images to quantify docks in glacial lakes. \textit{ISPRS Journal of Photogrammetry and Remote Sensing}. 81:60-69. ({\footnotesize\href{https://doi.org/10.1016/j.isprsjprs.2013.04.006}{10.1016/j.isprsjprs.2013.04.006}})

\item \textbf{Beck, M.W.}, Vondracek, B., and Hatch, L.K. 2013. Environmental clustering of lakes to evaluate performance of a macrophyte index of biotic integrity. \textit{Aquatic Botany}. 108:16-25. ({\footnotesize\href{https://doi.org/10.1016/j.aquabot.2013.02.003}{10.1016/j.aquabot.2013.02.003}})

\item \textbf{Beck, M.W.}, Claassen, A.H., and Hundt, P.J. 2012. Environmental and livelihood impacts of dams: Common lessons across development gradients that challenge sustainability. \textit{International Journal of River Basin Management}. 10(1):73-92. ({\footnotesize\href{http://dx.doi.org/10.1080/15715124.2012.656133}{10.1080/15715124.2012.656133}})

\item \textbf{Beck, M.W.}, Hatch, L.K., Vondracek, B., and Valley, R.D. 2010. Development of a macrophyte-based index of biotic integrity for Minnesota lakes. \textit{Ecological Indicators}. 10:968-979. ({\footnotesize\href{https://doi.org/10.1016/j.ecolind.2010.02.006}{10.1016/j.ecolind.2010.02.006}})

\item \textbf{Beck, M.W.}, and Hatch, L.K. 2009. A review of research on the development of lake indices of biotic integrity. \textit{Environmental Reviews}. 17:21-44. ({\footnotesize\href{https://doi.org/10.1139/A09-001}{10.1139/A09-001}})

\end{enumerate}

\vspace{\baselineskip} 
\centerline{\large{\textit{Technical Reports}}}

\begin{enumerate}

\item \textbf{Beck, M.W.}, Radabaugh, K.R., Flaherty-Walia, K.E. 2022. Recommendations for sampling effort of vegetation communities in the Critical Coastal Habitat Assessment. Tampa Bay Estuary Program Technical Report \#10-22. St. Petersburg, FL. ({\footnotesize\href{https://drive.google.com/file/d/1FAQFdw5oXunCQe-wGrwvnnOGTqaIHdFE/view?usp=sharing}{link}})

\item Schiff, K., \textbf{Beck, M.W.}, Fassman-Beck, E. 2021. North Orange County Municipal Separate Sewer System (MS4) Monitoring Evaluation. Technical Report \#1221. Southern California Coastal Water Research Project. Costa Mesa, CA. ({\footnotesize\href{https://ftp.sccwrp.org/pub/download/DOCUMENTS/TechnicalReports/1221_OCMonitoringEvaluation.pdf}{link}})

\item \textbf{Beck, M.W.}, Raulerson, G.E., Burke, M.C., Whalen, J., Scolaro, S., Sherwood, E.T. 2021. Tampa Bay Estuary Program: Data Management Standard Operating Procedures. Tampa Bay Estuary Program Technical Report \#12-21. St. Petersburg, FL. ({\footnotesize\href{https://drive.google.com/file/d/1vO4B8DJATgCSV1qOxZz-kN6Uj1BrgNsg/view?usp=sharing}{link}})

\item Meyers, S.D., Landry, S., \textbf{Beck, M.W.}, Luther, M.E. 2020. Understanding risk at wastewater treatment plants in Tampa Bay during extreme high-precipitation events. Tampa Bay Estuary Program Technical Report \#14-20. St. Petersburg, FL. ({\footnotesize\href{https://drive.google.com/file/d/1JTujiT7f7WUTF5r7__QqHwfq1oGmKj-s/view}{link}})

\item \textbf{Beck, M.W.}, Mazor, R.D. 2020. A decision framework for evaluating bioassessment samples and landscape models. Technical Report \#1115. Southern California Coastal Water Research Project. Costa Mesa, CA. ({\footnotesize\href{http://ftp.sccwrp.org/pub/download/DOCUMENTS/TechnicalReports/1115_CSCIDecisionFramework.pdf}{link}})

\item Taylor, J.B., Stein, E.D, \textbf{Beck, M.W.}, Flint, K., Kinoshita, A. 2019. Vulnerability of stream biological communities in Los Angeles and Ventura counties to climate change induced alterations of flow and temperature. Technical Report \#1084. Southern California Coastal Water Research Project. Costa Mesa, CA. ({\footnotesize\href{http://ftp.sccwrp.org/pub/download/DOCUMENTS/TechnicalReports/1084_ClimateChangeVulnerability.pdf}{link}})

\item \textbf{Beck, M.W.}, Mazor, R.D., Theroux, S.T., Schiff, K.C. 2019. The Stream Quality Index: A multi-indicator tool for enhancing environmental management communication. Technical Report \#1080. Southern California Coastal Water Research Project. Costa Mesa, CA. ({\footnotesize\href{http://ftp.sccwrp.org/pub/download/DOCUMENTS/TechnicalReports/1080_StreamQualityIndex.pdf}{link}})

\item \textbf{Beck, M.W.}, Kittleson, K., O'Connor, K. 2019. Analysis of the Juvenile Steelhead and Stream Habitat Database, Santa Cruz County, California: Web products and recommendations. Technical Report \#1082. Southern California Coastal Water Research Project. Costa Mesa, CA. ({\footnotesize\href{http://ftp.sccwrp.org/pub/download/DOCUMENTS/TechnicalReports/1082_SantaCruzSteelheads.pdf}{link}})

\item Afrooz, N., \textbf{Beck, M.W.}, Hale, T., McKee, L., Schiff, K. 2019. BMP performance monitoring data compilation to support reasonable assurance analysis. Technical Report \#1081. Southern California Coastal Water Research Project. Costa Mesa, CA. ({\footnotesize\href{http://ftp.sccwrp.org/pub/download/DOCUMENTS/TechnicalReports/1081_BMPPerformanceRAA.pdf}{link}})

\item Mazor, R.D., \textbf{Beck, M.W.}, Brown, J. 2018. 2017 report on the SMC Regional Stream Survey. Technical Report \#1029. Southern California Coastal Water Research Project. Costa Mesa, CA. ({\footnotesize\href{http://ftp.sccwrp.org/pub/download/DOCUMENTS/TechnicalReports/1029_2017SMCReport.pdf}{link}})

\item \textbf{Beck, M.W.}, Mazor, R.D., Stein, E.D., Maas, R., De Mello, D., Bram, D. 2017. Mapping of non-perennial and ephemeral streams in the Santa Ana region. Technical Report \#1012. Southern California Coastal Water Research Project. Costa Mesa, CA. ({\footnotesize\href{http://ftp.sccwrp.org/pub/download/DOCUMENTS/TechnicalReports/1012_MappingStreamsSantaAna.pdf}{link}})

\end{enumerate}

\vspace{\baselineskip} 
\centerline{\large{\textit{Software\footnote{software version numbers with .9000 indicate under development} and Websites}}}

\begin{enumerate}

\item \textbf{Beck, M.W.}, Villarroel, P., Padfield, D., Gaborini, L., von Maltzahn, N. 2023. rStrava: Functions to access data from Strava's v3 API, v1.2.0. R package. ({\footnotesize\href{https://cran.r-project.org/web/packages/rStrava/index.html}{link}})

\item \textbf{Beck, M.W.}, Hermann, M., Arriola, J., Najjar, R., Mcgillis, W. 2023. EBASE: Estuarine Bayesian Single-Station Estimation Method for Ecosystem Metabolism, v.0.0.0.9008. R package. ({\footnotesize\href{https://fawda123.github.io/EBASE/}{link}})

\item Payton, Q., Olsen, T., Weber, M., McManus, M., Kincaid, T., \textbf{Beck, M.W.} 2023. micromap: Linked micromap plots, v1.9.7. CRAN R package. ({\footnotesize\href{https://cran.r-project.org/web/packages/micromap/index.html}{link}})

\item \textbf{Beck, M.W.} 2023. SWMPr: An R package for the National Estuarine Research Reserve System, v2.4.3. CRAN R package. ({\footnotesize\href{http://fawda123.github.io/SWMPr}{link}})

\item \textbf{Beck, M.W.}, Carr, J. Wetherill, B., DiBona, P. 2023. MassWateR: Tools for QAQC and Exploratory Analysis of Massachusetts Surface Water Quality Data, v2.0.2. CRAN R package. ({\footnotesize\href{https://massbays-tech.github.io/MassWateR/}{link}})

\item Arriola, J.M., Hermann, M., \textbf{Beck, M.W.}. 2023. fwoxy: Forward model of estuary dissolved oxygen time series, v0.1. R package. ({\footnotesize\href{https://github.com/jmarriola/fwoxy}{link}}) 

\item \textbf{Beck, M.W.}, de Valpine, P., Murphy, R., Wren, I., Chelsky, A., Foley, M., Senn, D.B. 2022. wqtrends: Assess Water Quality Trends with Generalized Additive Models, v1.4.0. R package. ({\footnotesize\href{https://tbep-tech.github.io/wqtrends/}{link}})

\item \textbf{Beck, M.W.}, Schrandt, M., Wessel, M., Sherwood, E., Best, B. 2022. tbeptools: Data and Indicators for the Tampa Bay Estuary Program, v2.0.1. R package. ({\footnotesize\href{https://tbep-tech.github.io/tbeptools/}{link}})

\item \textbf{Beck, M.W.} 2022. ggord: A take on ordination plots with ggplot2, v1.1.7. R package. ({\footnotesize\href{https://fawda123.github.io/ggord/}{link}})

\item \textbf{Beck, M.W.} 2022. WtRegDO: Supplementary package to support the manuscript ``Improving estimates of ecosystem metabolism by reducing the effects of tidal advection on dissolved oxygen time series'', v1.0.0. R package. ({\footnotesize\href{http://github.com/fawda123/WtRegDO}{link}})

\item \textbf{Beck, M.W.} 2022. NeuralNetTools: Visualization and Analysis Tools for Neural Networks, v1.5.3. CRAN R package. ({\footnotesize\href{http://cran.r-project.org/web/packages/NeuralNetTools/}{link}})

\item \textbf{Beck, M.W.} 2020. WRTDStidal: Weighted regression for water quality evaluation in tidal waters, v1.1.3. CRAN R package. ({\footnotesize\href{https://cran.r-project.org/web/packages/WRTDStidal/index.html}{link}})

\item Bokde, N., \textbf{Beck, M.W.} 2019. imputeTestbench: Test bench for the comparison of imputation methods, v3.0.3. CRAN R package. ({\footnotesize\href{https://cran.r-project.org/web/packages/imputeTestbench/index.html}{link}})

\item \textbf{Beck, M.W.}, Kittleson, K., O'Connor, K. 2019. Juvenile Stream Survey Habitat Web (JSSH web). ({\footnotesize\href{https://sccwrp.shinyapps.io/jssh_web/}{link}})

\item \textbf{Beck, M.W.} 2018. SCCWRP R training. ({\footnotesize\href{https://sccwrp.github.io/SCCWRP_R_training/}{link}})

\item \textbf{Beck, M.W.} 2016. Jabbrev: Convert journal names from long to short formats, v0.0.1.9000. R package. ({\footnotesize\href{https://github.com/fawda123/Jabbrev}{link}})

\item \textbf{Beck, M.W.} 2016. Aggregation of SWMP parameters within/between reserves. Shiny Web Application. ({\footnotesize\href{http://beckmw.shinyapps.io/swmp_agg}{link}})

\item \textbf{Beck, M.W.} 2016. CTDplot: Plot CTD water quality data along an estuarine axis, v0.0.4.9000. R package. ({\footnotesize\href{https://github.com/fawda123/CTDplot}{link}})

\item O'Brien, T.D., \textbf{Beck, M.W.} 2015. Time series and data analysis information and tool resource for NERRS/SWMP. Website. ({\footnotesize\href{http://swmprats.net/}{link}})

\item \textbf{Beck, M.W.} 2015. Trends in SWMP parameters. Shiny Web Application. ({\footnotesize\href{http://beckmw.shinyapps.io/swmp_comp/}{link}})

\item \textbf{Beck, M.W.} 2015. Monthly and annual summary of SWMP parameters. Shiny Web Application. ({\footnotesize\href{http://beckmw.shinyapps.io/swmp_summary/}{link}})

\end{enumerate}

\vspace{\baselineskip} 
\centerline{\large{\textit{Outreach and Non-Technical Documents}}}

\begin{enumerate}

\item \textbf{Beck, M.W.} 2020. Automated reporting in Tampa Bay with Open Science. OpenScapes blog, National Center for Ecological Analysis \& Synthesis, University of California Santa Barbara. ({\footnotesize\href{https://www.openscapes.org/blog/2020/11/16/tampa-bay-reporting/}{link}}) 

\item \textbf{Beck, M.W.}, DeCicco, L. 2016. EGRET plotFlowConc using ggplot2. US Geological Survey, Office of Water Information Blog. ({\footnotesize\href{https://owi.usgs.gov/blog/plotFlowConc/}{link}})

\item \textbf{Beck, M.W.} 2012 - 2017. R is my friend - numerous posts. R-bloggers: R news and tutorials contributed by R bloggers. ({\footnotesize\href{https://beckmw.wordpress.com/}{link}})

\item \textbf{Beck, M.W.} 2012. 2nd annual bike across Minnesota fundraiser June 1\textsuperscript{st} through 4\textsuperscript{th}: Mission accomplished! The Cons Bio Blog, Conservation Biology Graduate Program, University of Minnesota. ({\footnotesize\href{https://consbioumn.wordpress.com/2012/06/25/2nd-annual-bike-across-minnesota-fundraiser-june-1st-through-4th-mission-accomplished/}{link}}).

\item \textbf{Beck, M.W.} 2012. Biking for bread. The Cons Bio Blog, Conservation Biology Graduate Program, University of Minnesota. ({\footnotesize\href{https://consbioumn.wordpress.com/2012/05/01/biking-for-bread/}{link}}).

\item \textbf{Beck, M.W.} 2011. Are values and conservation inseparable? The Cons Bio Blog, Conservation Biology Graduate Program, University of Minnesota. ({\footnotesize\href{https://consbioumn.wordpress.com/2011/09/21/are-values-and-conservation-inseparable/}{link}}).

\item \textbf{Beck, M.W.} 2011. What I do: An exercise in hyperbole. The Cons Bio Blog, Conservation Biology Graduate Program, University of Minnesota. ({\footnotesize\href{https://consbioumn.wordpress.com/2011/02/15/what-i-do-an-exercise-in-hyperbole/}{link}}).

\end{enumerate}

\vspace{\baselineskip} 
\centerline{\large{\textit{Conference Special Sessions}}}

\begin{enumerate}

\item Huddell, A., Ammerman, J., {\bf Beck, M.W.}, Hagy, J.D. 2022. Open Science for Collaborative Management of Aquatic Ecosystems. Joint Aquatic Sciences Meeting, Grand Rapids, Michigan. 

\item Zhang, Q., {\bf Beck, M.W.}, Bertani, B., Keisman, J., Preheim, S.P., Neeley, A.R.. 2021. Developing New Insights from Environmental Data Through Innovative Analysis Approaches. Coastal and Estuarine Research Federation Virtual Meeting.

\item Zhang, Q., Murphy, R., {\bf Beck, M.W.}, Keisman, J. 2019. Innovative Approaches for Estuarine/Watershed Data Analysis, Mining, and Visualization. Coastal and Estuarine Research Federation biennial meeting, Mobile, AL.

\end{enumerate}

\vspace{\baselineskip} 
\centerline{\large{\textit{Invited Presentations}}}

\begin{enumerate}

\item {\bf Beck, M.W.} 2023. Piney Point, red tide, and Twitter: A Tampa Bay story. University of South Florida, Integrative Biology Departmental Seminar, Tampa, FL. ({\footnotesize\href{https://docs.google.com/presentation/d/1yfXyuxaXU2CPLWnhtApxDpyS_xuhm9fh29jPpw6MaHE/edit?usp=share_link}{link}})

\item {\bf Beck, M.W.} 2022. Better nutrient management through Open Science: A collaboration of tales from California to Florida. \textit{Oral presentation}. USEPA Nutrient Scientific Technical Exchange Partnership and Support Program (virtual).  

\item {\bf Beck, M.W.}, Sherwood, E.T., Burke, M.C., Raulerson, G.E. 2022. How the Tampa Bay Estuary Program Uses Open Science to Work Smarter, Not Harder. \textit{Oral presentation}. Joint Aquatic Sciences Meeting (virtual attendee), Grand Rapids, Michigan. 

\item {\bf Beck, M.W.}, Burke, M.C., Raulerson, G.E., Scolaro, S.R., Sherwood, E.T., Whalen, J. 2021. Coordinated monitoring of the Piney Point wastewater discharge into Tampa Bay. \textit{Oral presentation}. Coastal and Estuarine Research Federation Virtual Meeting.

\item Murphy, R., Keisman, J., Perry, E., Harcum, J., Leppo, E., {\bf Beck, M.W.}. 2021. Identifying, visualizing, and explaining estuarine water quality changes with Generalized Additive Models. \textit{Oral presentation}. Coastal and Estuarine Research Federation Virtual Meeting.

\item {\bf Beck, M.W.},  Sherwood, E.T., Henkel, J.R., Dorans, K., Ireland, K., Varela, P. 2021. Spatial analysis of the cumulative effects of pollution and restoration on water quality and seagrass recovery.  National Conference on Ecosystem Restoration Virtual Meeting.

\item {\bf Beck, M.W.}, Wren, I., Murphy, R.R., de Valpine, P., Senn, D. 2019. Tracking San Francisco Bay water quality using generalized additive models in an R Shiny framework. \textit{Oral presentation}. Coastal and Estuarine Research Federation biennial meeting, Mobile, AL.

\item {\bf Beck, M.W.} 2018. Actionable bioassessment data: Using visualization to bridge the research - management divide. \textit{Oral presentation}. California Aquatic Bioassessment Workgroup, Davis, CA.

\item Mazor, R., {\bf Beck, M.W.}, Ode, P., Johnson, S., Wisenbaker, K., Westfall, J., Markle, P. 2018. Supporting management decisions with a landscape model to predict biotic condition in California. \textit{Oral presentation}. Society for Freshwater Science Annual Meeting, Detroit, MI.

\item {\bf Beck, M.W.}, Sherwood, E., Dorans, K., Henkel, J.R., Ireland, K., Varela, P. 2018. Use of open science to inform restoration projects in estuaries: A Tampa Bay example. American Water Resources Association GIS Spring Specialty Conference, Orlando, FL. 

\item {\bf Beck, M.W.}, Senn, D., Novick, E., Bresnahan, B., Hagy, J.D., Jabusch, T. 2016. Four decades of water quality changes in the upper San Francisco Estuary. \textit{Poster presentation}. Interagency Ecological Program Annual Workshop, Folsom, CA.

\item {\bf Beck, M.W.}, Hagy, J.D. 2015. Adaptation of a weighted regression approach to evaluate water quality trends in an estuary. \textit{Oral presentation}. Coastal and Estuarine Research Federation Biennial Conference, Portland, OR.

\item {\bf Beck, M.W.}, O'Brien, T.D. 2015. SWMPrats: A community of practice for NERRS data analysis. \textit{Oral presentation}. NERRS research coordinators virtual meeting, USEPA NHEERL ORD, Gulf Breeze, FL.

\item {\bf Beck, M.W.} 2014. The search for truth in numbers: Quantitative approaches for evaluating trends in water quality data. \textit{Oral presentation}. Biology Department seminar (PCB 3930/4922/5924), University of West Florida, Pensacola, FL. 

\item {\bf Beck, M.W.}, Vondracek, B., Wilson, B., and Hatch, L.K. 2013. Understanding indicators of lake health and the utility of a plant-based index. \textit{Oral presentation}. Gulf Ecology Division seminar series, USEPA NHEERL ORD, Gulf Breeze, FL.

\item {\bf Beck, M.W.}, Vondracek, B., Hatch, L.K., and Wilson, B. 2013. Minnesota macrophytes: Linking aquatic plants, lake integrity, and human activities. \textit{Poster presentation}. Doctoral Research Showcase, University of Minnesota, Minneapolis, MN.

\item {\bf Beck, M.W.}, Vondracek, B., and Hatch, L.K. 2012. Between and within lake responses of macrophyte richness metrics to shoreline development. \textit{Oral presentation}. MN Department of Natural Resources Fall research meeting, Itasca State Park, MN.

\item {\bf Beck, M.W.}, Vondracek, B., Hatch, L.K. 2011. Biological assessment of aquatic macrophytes in Midwest glacial lakes. \textit{Oral presentation}. University of Minnesota Water Resources Graduate Program student symposium, Stillwater, MN.

\item {\bf Beck, M.W.} 2009. Perspectives on sustaining Minnesota's lakes: Biological monitoring, IBIs, and stressor identification. \textit{Oral presentation}. University of Minnesota Water Resources Graduate Program seminar, St. Paul, MN.

\item {\bf Beck, M.W.} 2009. The Minnesota macrophyte IBI for lake assessment. \textit{Oral presentation}. Habitat research meeting, Minnesota Department of Natural Resources, St. Paul, MN.

\item {\bf Beck, M.W.}, Hatch, L.K., Vondracek, B., Valley, R.D. 2009. Development of a macrophyte-based index of biotic integrity for Minnesota lakes. \textit{Oral presentation}. Aquatic plant assessment meeting, Minnesota Pollution Control Agency, St. Paul, MN.

\item {\bf Beck, M.W.}, Hatch, L.K., Vondracek, B., Valley, R.D. 2009. Development of a macrophyte-based index of biotic integrity for Minnesota lakes. \textit{Oral presentation}. Fisheries research meeting, Minnesota Department of Natural Resources, Cloquet, MN.

\end{enumerate}

\vspace{\baselineskip} 
\centerline{\large{\textit{Contributed Presentations}}}

\begin{enumerate}

\item Flaherty-Walia, K., {\bf Beck, M.W.}, Schrandt, M., Sherwood, E. Emerging signs of ecological stress in Tampa Bay reinvigorates partnerships to preserve and restore seagrass habitats in support of healthy nekton communites. Florida chapter of the Am. Fisheries Society annual meeting, St. Augustine, FL.

\item {\bf Beck, M.W.}, Sherwood, E.T., Burke, M.C., Scolaro, S., Flaherty-Walia, K. 2023. Using open science to address current and future threats to the health of Tampa Bay. \textit{Oral presentation}. National Water Quality Monitoring Conference biennial meeting, virginia Beach, VA.

\item Sherwood, E.T., Burke, M.C., {\bf Beck, M.W.}, Janicki, T., Pribble, R., Lopez, C., Shankar, S., Kaminski, S. 2022. Tampa Bay restoration and \textit{Pyrodinium bahamense} bloom dynamics: Filling knowledge gaps to enhance recovery. \textit{Oral presentation}. Gulf of Mexico Conference.

\item Sherwood, E.T., {\bf Beck, M.W.}, Burke, M.C., Raulerson, G.e., Scolaro, S.R., Whalen, J. 2021. Understanding and communicating the impacts of the Piney Point wastewater discharge into Tampa Bay. \textit{Oral presentation}. Coastal and Estuarine Research Federation Virtual Meeting.

\item Arriola, J.M., {\bf Beck, M.W.}, Bourque, E., Caffrey, J., Herrmann, M., Najjar, R.G. 2021. Improving estuarine net ecosytem production estimates with detiding: Revisiting twenty years of data at Apalachicola. \textit{Oral presentation}. Coastal and Estuarine Research Federation Virtual Meeting.

\item Wessel, M., \textbf{Beck, M.W.}, Leverone, J., Sherwood, E.T., Hecker, J., Raulerson, G., Janicki, T. 2020. Tidal tribulations: Overcoming hurdles in identifying a stressor-response relationship to develop a water quality management framework for Southwest Florida tidal creeks. American Water Resources Association Virtual Conference. 

\item \textbf{Beck, M.W.}, Wessel, M., Leverone, J., Sherwood, E.T., Raulerson, G., Janicki, T. 2020. Using an applied open source toolbox for tidal creek assessment in Southwest Florida. American Water Resources Association Virtual Conference. 

\item Bednar\v{s}ek, N., Feely, R.A., Alin, S.R., Siedlecki, S.A., \textbf{Beck, M.W.}, Calosi, P., Pelletier, G., Saenger, C., Spicer, J.I. 2019. Modelling carapace dissolution and growth under present-day low pH gradients in larval Dungeness crab.  Coastal and Estuarine Research Federation biennial meeting, Mobile, AL.

\item Sherwood, E., \textbf{Beck, M.W.}, Henkel, J.R., Dorans, K., Ireland, K., Varela, P. 2019. Assessing the cumulative effects of restoration activities on improving water quality in Tampa Bay, Florida.  Coastal and Estuarine Research Federation biennial meeting, Mobile, AL.

\item \textbf{Beck, M.W.}, O'Hara, C., Stewart Lowndes, J., Mazor, R.D., Theroux, S.T., Gillett, D.J., Lane, B., Gearheart, G. 2019. The importance of open science for biological assessment. \textit{Oral presentation}. Society for Freshwater Science Annual Meeting, Salt Lake City, UT.

\item {\bf Beck, M.W.}, Sherwood, E., Kirsten, D., Ireland, K., Varela, P., Henkel, J. 2019. Use of open science to inform restoration projects in estuaries: A Tampa Bay example. \textit{Oral presentation}. Gulf of Mexico Oil Spill \& Ecosystem Science Conference, New Orleans, LA.

\item {\bf Beck, M.W.}, Kittleson, K., O'Connor, K. 2018. Customized web-based exploration of a long-term fisheries monitoring program. \textit{Ignite presentation}. California Estuarine Research Society, Long Beach, CA.

\item {\bf Beck, M.W.}, Sutula, M., Howard, M., Stein, E. 2018. Landscape scale risk assessment of cyanobacteria blooms in California lakes. \textit{Oral presentation}. Society for Freshwater Science Annual Meeting, Detroit, MI.

\item Wan, Y., {\bf Beck, M.W.}. 2017. Comparative analysis of long-term chlorophyll data with Generalized Additive Model. \textit{Oral presentation}. Coastal and Estuarine Research Federation biennial meeting, Providence, RI.

\item {\bf Beck, M.W.}, Hagy, J.D., Le, C. 2017. Quantifying seagrass light requirements using an algorithm to spatially resolve depth of colonization. \textit{Oral presentation}. Coastal and Estuarine Research Federation biennial meeting, Providence, RI.

\item {\bf Beck, M.W.}, O'Brien, T.D., Cressman, K., St. Laurent, K., Eslinger, D. 2016. SWMPrats.net: A web-based resource for exploring SWMP data. \textit{Poster presentation}. National Estuarine Research Reserve System annual meeting, Williamsburg, VA. 

\item {\bf Beck, M.W.}, Hagy, J.D., Le, C. 2016. Quantifying seagrass light requirements using an algorithm to spatially resolve depth of colonization. \textit{Oral presentation}. Gulf Estuarine Research Society biennial meeting, Pensacola Beach, FL.

\item {\bf Beck, M.W.}, O'Brien, T.D., Bundy, M.H. 2016. Integrated analysis tools for the NERRS System-Wide Monitoring Program Data. \textit{Oral presentation}. National Water Quality Monitoring Conference biennial meeting, Tampa, FL.

\item Cressman, K., Caffrey, J., {\bf Beck, M.W.}, Griffin, C. 2016. Understanding long-term changes by linking monthly chlorophyll measurements to high-frequency water quality data. \textit{Oral presentation}. National Water Quality Monitoring Conference biennial meeting, Tampa, FL.

\item Senn, D., Otten, T., Stewart, R., Bresnahan, P., {\bf Beck, M.W.}, Hagy, J.D., Guerin, M., Jabusch, T., Kendall, C., Novick, E., Trowbridge, P., Young, M. 2016. Probing archived samples and historic data for improved understanding of nutrient cycling and ecosystem response in the Bay-Delta. \textit{Oral presentation}. Interagency Ecological Program Annual Workshop, Folsom, CA.

\item Hagy, J.D., {\bf Beck, M.W.}, Murrell, M.C. 2015. Improving estimates of ecosystem metabolism by reducing effects of tides on DO time series. \textit{Oral presentation}. Coastal and Estuarine Research Federation Biennial Meeting, Portland, OR. 

\item Bundy, M.H., {\bf Beck, M.W.}, O'Brien, T.D. 2015. Synthesis of NERRS System-Wide Monitoring Program data: Using integrated analysis tools to detect trends. \textit{Oral presentation}. Coastal and Estuarine Research Federation Biennial Meeting, Portland, OR.

\item {\bf Beck, M.W.} 2015. SWMPr: An R package for estuarine water quality time series. \textit{Oral presentation}. USEPA R User Group meeting, USEPA NHEERL ORD, Gulf Breeze, FL.

\item {\bf Beck, M.W.}, Hagy, J.D., Murrell, M.C. 2014. A novel approach for evaluation of water quality trends in Gulf Coast estuaries. \textit{Oral presentation}. Bays and Bayous Symposium, Mobile, AL.

\item Murrell, M.C., Hagy, J.D., Aukamp, J., {\bf Beck, M.W.}, Beddick, D., Craven, G., Duffy, A., Jarvis, B.M., Marcovich, M., Yates, D., Caffrey, J. 2014. Environmental drivers of ecosystem and plankton metabolism in Pensacola Bay, Florida. Bays and Bayous Symposium, Mobile, AL. 

\item Hagy, J.D., Jarvis, B.M., Murrell, M.C., {\bf Beck, M.W.}. 2014. Modeling remediation of aquatic life impacts of episodic and diel cycling hypoxia via nutrient loading rate reduction. Bays and Bayous Symposium, Mobile, AL. 

\item {\bf Beck, M.W.} 2014. ggplot2 redux. \textit{Oral presentation}. USEPA R User Group meeting, USEPA NHEERL ORD, Gulf Breeze, FL.

\item {\bf Beck, M.W.}, Tomcko, C.M., Valley, R.D., and Staples, D.F. 2014. Analysis of macrophyte indicator variation as a function of sampling, temporal, and stressor effects. \textit{Poster presentation}. Joint Aqautic Sciences Meeting, Portland, OR.

\item {\bf Beck, M.W.}, Hagy, J.D. 2014. Adaptation of a weighted regression approach to evaluate water quality trends in {T}ampa {B}ay, {F}lorida. \textit{Oral presentation}. National Water Quality Monitoring Conference biennial meeting, Cincinnati, OH.

\item {\bf Beck, M.W.}. 2014. The nuts and bolts of Sweave/Knitr for reproducible research. \textit{Oral presentation}. USEPA R User Group meeting, USEPA NHEERL ORD, Gulf Breeze, FL.

\item {\bf Beck, M.W.}, Vondracek, B., Hatch, L.K., and Wilson, B. 2013. Evaluating the utility of a plant-based index of lake condition using neural networks. \textit{Oral presentation}. Ecological Society of America annual meeting, Minneapolis, MN.

\item {\bf Beck, M.W.}, Vondracek, B., Wilson, B., and Hatch, L.K. 2013. Understanding indicators of lake health and the utility of a plant-based index. \textit{Oral presentation}. Conservation Biology Graduate Program seminar, University of Minnesota, St. Paul, MN.

\item Vondracek, B., Koch, J.D., and {\bf Beck, M.W.} 2013. A comparison of survey methods to evaluate macrophyte index of biotic integrity performance in Minnesota lakes. \textit{Oral presentation}. Minnesota chapter of the Am. Fisheries Society annual meeting, St. Cloud, MN.

\item {\bf Beck, M.W.}, Vondracek, B., Wilson, B., and Hatch, L.K. 2013. Understanding indicators of lake health and the utility of a plant-based index. \textit{Oral presentation}. Conservation Biology Graduate Program seminar, University of Minnesota, St. Paul, MN.

\item {\bf Beck, M.W.} 2012. Effects of shoreline development on aquatic macrophytes: Summary of analyses. \textit{Class lecture}. Ecological Modeling (FW 8200), University of Minnesota, St. Paul, MN.

\item {\bf Beck, M.W.}, Vondracek, B., Hatch, L.K., and Vinje, J. 2012. Image analysis techniques to evaluate effects of lakeshore development on aquatic habitat. \textit{Oral and poster presentation}. Am. Fisheries Society annual meeting, St. Paul, MN.

\item {\bf Beck, M.W.}, Vondracek, B., Hatch, L.K., and Vinje, J. 2012. Image analysis techniques to evaluate effects of lakeshore development on aquatic habitat. \textit{Oral presentation}. Water Resources Conference, St. Paul, MN.

\item {\bf Beck, M.W.}, Vondracek, B., Hatch, L.K., 2012. Identifying covariates of a lake assessment index to improve biological assessment. \textit{Oral presentation}. Minnesota chapter of Society for Conservation Biology annual meeting, St. Paul, MN.

\item {\bf Beck, M.W.}, Vondracek, B., Hatch, L.K. 2011. Image analysis techniques to evaluate effects of nearshore lake development on aquatic macrophytes. \textit{Oral presentation}. Midwest Fish and Wildlife Conference, Des Moines, IA.

\item {\bf Beck, M.W.} 2011. R is my friend: Or how I learned to stop worrying and love the machine. \textit{Oral presentation}. Conservation Biology Graduate Program breakfast seminar, University of Minnesota, St.Paul, MN.

\item {\bf Beck, M.W.}, Vondracek, B., Hatch, L.K. 2011. Image analysis techniques to evaluate effects of nearshore lake development on aquatic macrophytes. \textit{Poster presentation}. Water Resources Conference, St. Paul, MN.

\item {\bf Beck, M.W.}, Vondracek, B., Hatch, L.K. 2011. Image analysis techniques to evaluate effects of nearshore lake development on aquatic macrophytes. \textit{Oral presentation}. Am. Fisheries Society annual meeting, Seattle, WA.

\item {\bf Beck, M.W.}, Vondracek, B., Hatch, L.K. 2011. Image analysis techniques to evaluate effects of nearshore lake development on aquatic macrophytes.\textit{Oral presentation}. Minnesota chapter of the Am. Fisheries Society annual meeting, Sandstone, MN.

\item {\bf Beck, M.W.}, Hatch, L.K., Vondracek, B., Valley, R.D. 2010. Development of a macrophyte-based index of biotic integrity for Minnesota lakes. \textit{Poster presentation}. Midwest Fish and Wildlife Conference, Minneapolis, MN.

\item Koch, J.D., {\bf Beck, M.W.}, Carlin, J. 2010. A comparison of survey methods to evaluate macrophyte index of biotic integrity performance in Minnesota lakes. \textit{Poster presentation}. Midwest Fish and Wildlife Conference, Minneapolis, MN.

\item {\bf Beck, M.W.}, Hatch, L.K., Vondracek, B., Valley, R.D. 2010. Development of a macrophyte-based index of biotic integrity for Minnesota lakes. \textit{Oral presentation}. International Congress for Conservation Biology, Edmonton, Alberta.

\item {\bf Beck, M.W.}, Vondracek, B., Hatch, L.K. 2010. Assessing the health of Minnesota's lakes using indices of biotic integrity. \textit{Oral presentation}. Joint meeting of the Minnesota chapters of Am. Fisheries Society, Society for Conservation Biology, and The Wildlife Society, Nisswa, MN.

\item {\bf Beck, M.W.}, Hatch, L.K., Vondracek, B., Valley, R.D. 2009. Development of a macrophyte-based index of biotic integrity for Minnesota lakes. \textit{Oral presentation}. Water Resources Conference, St. Paul, MN.

\item {\bf Beck, M.W.}, Hatch, L.K., Vondracek, B., Valley, R.D. 2009. Development of a macrophyte-based index of biotic integrity for Minnesota lakes. \textit{Poster presentation}. Land Conservation and Clean Water Summit, Chaska, MN.

\item {\bf Beck, M.W.}, Hatch, L.K., Vondracek, B., Valley, R.D. 2009. Development of a macrophyte-based index of biotic integrity for Minnesota lakes. \textit{Oral presentation}. Symposium for best student paper, Am. Fisheries Society annual meeting, Nashville, TN.

\item {\bf Beck, M.W.}, Hatch, L.K., Vondracek, B., Valley, R.D. 2009. Development of a macrophyte-based index of biotic integrity for Minnesota lakes. \textit{Oral presentation}. Minnesota chapter of the Society for Conservation Biology annual meeting, St. Paul, MN.

\item {\bf Beck, M.W.}, Hatch, L.K., Vondracek, B., Valley, R.D. 2008. Development of a macrophyte-based index of biotic integrity for Minnesota lakes. \textit{Poster presentation}. Water Resources Conference, St. Paul, MN.

\item {\bf Beck, M.W.} 2008. The index of biotic integrity: A review. \textit{Class lecture}. Water quality and natural resources (ESPM 4061), University of Minnesota, St. Paul, MN.

\item {\bf Beck, M.W.}, Hatch, L.K., Vondracek, B., Valley, R.D. 2008. Development of a macrophyte-based index of biotic integrity for Minnesota lakes. \textit{Oral presentation}. Conservation Biology Graduate Program seminar, University of Minnesota, St. Paul, MN.

\item Hatch, L.K., {\bf Beck, M.W.}, Vondracek, B. 2007. Ecological assessment method development for Minnesota lakes. \textit{Oral presentation}. North Am. Lake Management Society annual meeting, Orlando, FL.

\end{enumerate}

\sectitle{Grants, Contracts, Awards, and Honors}

San Franciso Estuary Institude Contract No. 1571 {\bf \$12000} \hfill {\bf Mar. 2023} \\
USEPA ORD, Level III Science and Technological Achievement Award \hfill {\bf Jun. 2022} \\
San Franciso Estuary Institude Contract No. 1571 {\bf \$9600} \hfill {\bf Jan. 2022} \\
Commonwealth of Massachusetts {\bf \$37856} \hfill {\bf Nov. 2021} \\
University of South Florida PO P000014147 {\bf \$4950} \hfill {\bf Oct. 2021} \\
Penn State University Contract No. OCE 19-24559 {\bf \$42441} \hfill {\bf Jan. 2020} \\
San Francisco Estuary Institute Contract No. 1456 {\bf \$18952} \hfill {\bf Jan. 2020} \\
Southern California Coastal Water Research Project Task Order No. 17171 {\bf \$21893} \hfill {\bf Jan. 2020} \\
Janicki Environmental, Inc. TBEP Tech Support Contract for Work Order 1 {\bf \$10600} \hfill {\bf May 2019} \\
Aquatic Bioassay and Consulting Labs, Inc. PR No. SCC0419.003 {\bf \$99488} \hfill {\bf Apr. 2019} \\
County of Santa Cruz, California PR No. BR410375 {\bf \$5000} \hfill {\bf Oct. 2018} \\
San Francisco Estuary Institute Contract No. 1358 {\bf \$10000} \hfill {\bf Aug. 2018} \\
US EPA Gulf Ecology Division Green Award \hfill {\bf May 2017} \\
ORISE post-doctorate research highlights \hfill {\bf Feb. 2015} \\
LaBounty Best Paper Award nomination, N. Am. Lake Management Society \hfill {\bf Oct. 2014} \\
Interdisciplinary Doctoral Fellowship, University of Minnesota {\bf \$22500} \hfill {\bf Sep. 2012} \\
Best student paper award, MN chapter for Society of Conservation Biology \hfill {\bf Mar. 2012} \\
Carolyn Crosby Fellowship, University of Minnesota {\bf \$1000} \hfill {\bf Jan. 2012} \\
Best student paper award, MN chapter of Am. Fisheries Society {\bf \$200} \hfill {\bf Feb. 2011} \\
Fenske Memorial Award finalist, Midwest Fish and Wildlife Conference \hfill {\bf Dec. 2010} \\
Travel award, Strategic Env. Research and Development Program {\bf \$700} \hfill {\bf Jul. 2010} \\
Travel award, Graduate and Professional Student Assembly {\bf \$200} \hfill {\bf Jul. 2010} \\
Finalist in best student paper symposium, Am. Fisheries Society \hfill {\bf Sep. 2009} \\
Conservation Biology Graduate Program summer fellowship {\bf \$3500} \hfill {\bf Jun. 2008} \\
Phi Beta Kappa Honor Society \hfill {\bf Dec. 2006} \\
Dean's list for academic excellence duration of undergraduate education \hfill {\bf 2004 - 2006} \\
Florida Bright Futures undergraduate scholarship {\bf \$5250} \hfill {\bf 2003 - 2006} 

\sectitle{Professional Society Membership}

Coastal and Estuarine Research Federation \hfill {\bf Sep. 2013 - Present} \\
Society for Conservation Biology \hfill {\bf Feb. 2008 - Present} \\
Society for Freshwater Science \hfill {\bf Apr. 2018 - Present}

\sectitle{Technical Training and Additional Experience}

Expert knowledge in open science tools and their application, including R, RStudio, \\
\hspace{0.3in}R Shiny, Git, GitHub, \LaTeX, Sweave, knitr \\
Workshop participant, Master R Developer, San Francisco, CA \\
Webinar participant, The Grammar and Graphics of Data Science \\
Workshop participant, New Tools for Water Quality Data Access and Trend \\ \hspace{0.3in} Analysis. An Overview of the USGS R Packages: dataRetrieval and EGRET\\
Workshop participant, An Introduction to the Fundamentals of Linear \\
\hspace{0.3in}Quantile Regression in R \\
Workshop participant, Aquatic Plant Identification, Itasca State Park, MN \\
SCUBA open water and advanced Nitrox certification \\
CPR and first aid certification

\end{document}