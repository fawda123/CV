\documentclass[letterpaper,12pt]{article} 
% \usepackage{times}
\usepackage[top=1in,bottom=1in,left=1in,right=1in]{geometry}
\usepackage{graphics}
\usepackage{booktabs}
\usepackage{fancyhdr}
\usepackage{xcolor}
\usepackage{enumitem}
\usepackage{url}
\usepackage{datetime}
\usepackage[colorlinks=true,urlcolor=blue,citecolor=blue,linkcolor=blue]{hyperref}

\newcommand{\sectitle}[1]{\vspace{\baselineskip} \centerline{\large{\textit{#1}}}} 
\newdateformat{mydate}{\monthname[\THEMONTH] \THEYEAR}
\urlstyle{same}

\setlist{noitemsep,topsep=0pt,parsep=0pt,partopsep=0pt,leftmargin=*}
\setlength{\parindent}{0pt}
\setlength{\parskip}{\baselineskip}%
\pagestyle{fancy}
\fancyhf{}
\fancyfoot[R]{\textcolor{gray}{Marcus W. Beck, CV page \thepage}}
\fancyfoot[L]{\textcolor{gray}{\mydate\today}}
\renewcommand{\headrulewidth}{0pt}
% \renewcommand{\rmdefault}{ptm}

\begin{document}

\raggedright

\LARGE
\centerline{{\bf Dr. Marcus William Beck}}
\normalsize
\textit{USEPA NHEERL Gulf Ecology Division \hfill Office: 850-934-2480 \\
1 Sabine Island Drive \hfill Cell: 352-871-6795 \\
Gulf Breeze, FL 32561 \hfill Email: beck.marcus@epa.gov}
\vspace{4pt}
\hrule
\vspace{2pt}
\hrule
\vspace{4pt}

\sectitle{INTERESTS}

Water quality analysis $\bullet$ data synthesis, visualization, and interpretation $\bullet$ reproducible research and open-source analysis $\bullet$ multivariate statistics $\bullet$ aquatic ecology $\bullet$ environmental indicators $\bullet$ water quality policy

\sectitle{EDUCATION}

{\bf Ph.D., Conservation Biology, Fisheries and Aquatic Biology track} \hfill {\bf Jun. 2013} \\
{\bf Statistics minor} \\
University of Minnesota, Twin Cities, MN \\
Advisors: Drs. Bruce Vondracek, Lorin Hatch $\bullet$ GPA 3.94 

{\bf M.Sc., Conservation Biology, Fisheries and Aquatic Biology track} \hfill {\bf Jun. 2009} \\
University of Minnesota, Twin Cities, MN \\
Advisor: Dr. Lorin Hatch $\bullet$ GPA 4.00

{\bf B.Sc., Zoology} \textit{summa cum laude} \hfill {\bf Dec. 2006} \\
University of Florida, Gainesville, FL \\
Advisor: Dr. Krista McCoy $\bullet$ GPA 3.81

{\bf A.A., Zoology} \hfill {\bf Apr. 2004} \\
Santa Fe Community College, Gainesville, FL \\
GPA 3.87

\sectitle{EXPERIENCE}

{\bf Post-Doctorate Research Fellow}, USEPA \hfill {\bf Jul. 2013 - Present}\\
\textit{Supervisor}: Dr. Jim Hagy \\
Held post-doctorate research fellowship position in Oak Ridge Institute for Science and Education at USEPA Gulf Ecology Division, Ecosystem Dynamics and Effects Branch $\bullet$ Developed and applied statistical models of water quality data in coastal waters to guide development and implementation of standards

{\bf Workshop Co-Instructor}, NOAA \hfill {\bf Jul. 2014 - Nov. 2014}\\
Developed data management and analysis tools for the System-Wide Monitoring Program (SWMP) of the National Estuarine Research Reserve System to be co-taught at the annual training meeting $\bullet$ Developed software package for SWMP data to support workshop content

{\bf Research Fellow}, University of Minnesota \hfill {\bf Sep. 2012 - Jun. 2013} \\
\textit{Supervisor}: Dr. Bruce Wilson \\
Developed and analyzed biological indices with agency and citizen monitoring datasets for the assessment of lake condition in support of resource management $\bullet$ Prepared progress reports and public presentations to communicate results and provide management recommendations

{\bf Short Course Co-Instructor}, University of Minnesota \hfill {\bf May 2013} \\
\textit{Supervisor}: Dr. Susan Galatowitsch \\
Co-instructed two short courses on basic data analysis that were part of a larger workshop focused on ecosystem restoration $\bullet$ Provided instruction using interactive lectures and take-home material as an introduction to software R

{\bf Research Assistant}, University of Minnesota \hfill {\bf Sep. 2007 - Sep. 2012} \\
\textit{Supervisor}: Dr. Bruce Vondracek \\
Developed an index of biotic integrity using aquatic macrophytes to determine the ecological health of Minnesota lakes $\bullet$ Developed remote sensing and image analysis techniques to quantify land use stressors $\bullet$ Assisted researchers at the Minnesota Department of Natural Resources to develop methods for long-term lake monitoring program

{\bf Primary Instructor}, University of Minnesota \hfill {\bf Jan. 2012 - May 2012} \\
Instructed graduate and undergraduate students on fundamental topics in fisheries ecology and management (FW 5604, spring 2012)

{\bf Undergraduate Advisor}, University of Minnesota \hfill {\bf Mar. 2009 - May 2011} \\
Advised undergraduate projects that examined survey designs for biological assessment of aquatic plants and use of fish indices to assess lake condition 

{\bf Intern}, Minnesota Department of Natural Resources \hfill {\bf Jun. 2009 - Aug. 2009} \\
\textit{Supervisor}: Timothy Cross \\
Assisted in collection, preparation, and analysis of data in support of lake monitoring research

{\bf Teaching Assistant}, University of Minnesota \hfill {\bf Sep. 2008 - Dec. 2008} \\
\textit{Supervisor}: Dr. Joseph Magner \\
Held teaching assistant position for ESPM 4061/5061 (Water Quality and Natural Resources, 3cr.)

{\bf Creel Clerk}, Florida Fish and Wildlife Commission \hfill {\bf Jan. 2007 - May 2007} \\
\textit{Supervisor}: Eric Nagid \\
Conducted creel surveys of anglers for catch and effort of black crappie (\textit{Pomoxis nigromaculatus}) $\bullet$ Conducted laboratory analyses to support fisheries monitoring program

{\bf Undergraduate Researcher}, University of Florida \hfill {\bf Jan. 2006 - Dec. 2006} \\
\textit{Supervisors}: Dr. Krista McCoy, Dr. Colette St. Mary \\
Investigated effects of pesticides on green tree frogs (\textit{Hyla cinerea}) $\bullet$ Performed slide histology and tissue staining

{\bf Biological Technician}, US Geological Survey \hfill {\bf summer 2006} \\
\textit{Supervisor}: Noel Burkhead \\
Assisted with investigation of red shiner (\textit{Cyprinella lutrensis}) invasion $\bullet$ Collected mortality data on lab-reared fishes $\bullet$ Conducted stream sampling

\sectitle{SERVICE}

{\bf Reviewer}\hfill \\
\textit{Acta Oecologica}, 2015 $\bullet$ \textit{Aquatic Botany}, 2011, 2014 $\bullet$ \textit{Development in Practice}, 2012, 2013 $\bullet$ \textit{Ecological Indicators}, 2010 $\bullet$ \textit{Ecological Research}, 2012 $\bullet$ Elsevier book chapter, 2014 $\bullet$ \textit{Environmental Management}, 2010 $\bullet$ \textit{Estuaries and Coasts}, 2014, 2015 $\bullet$ \textit{Hydrobiologia}, 2012 $\bullet$ \textit{International Journal of Tropical Biology and Conservation}, 2013 $\bullet$ \textit{International Journal of River Basin Management}, 2013 $\bullet$ \textit{Journal of Applied Phycology}, 2015 $\bullet$ \textit{Journal of the North American Benthological Society}, 2011 $\bullet$ \textit{Lake and Reservoir Management}, 2013, 2014 $\bullet$ \textit{Limnology \& Oceanography}, 2013 $\bullet$ \textit{Marine Ecology}, 2014 $\bullet$ \textit{Marine Pollution Bulletin}, 2014 $\bullet$ Minnesota Department of Natural Resources, 2011 $\bullet$ \textit{North American Journal of Fisheries Management}, 2010, 2011, 2012 $\bullet$ \textit{Polar Biology}, 2011 $\bullet$ \textit{Remote Sensing}, 2014 $\bullet$ \textit{Science of the Total Environment}, 2011 $\bullet$ USEPA Office of Research and Development, 2014

{\bf Graduate student representative}, University of Minnesota \hfill {\bf Jan. 2013 - May 2013} \\
Served as graduate student representative and voting member for faculty meetings in the Fisheries and Wildlife Department

{\bf Planning Committee Co-Chair}, Am. Fisheries Society \hfill {\bf Jun. 2010 - Aug. 2012} \\
Served as student activities committee co-chair for 2012 annual meeting of the Am. Fisheries Society $\bullet$ Organized student social, career fair, student colloquium for 350 students

{\bf Fundraising Coordinator}, University of Minnesota \hfill {\bf spring 2011, spring 2012} \\
Committee chair for Conservation Biology Graduate Program fundraiser in spring 2012, active fundraiser in spring 2011 $\bullet$ Raised approximately \$6000 in funds for student travel

{\bf Seminar Coordinator}, University of Minnesota \hfill {\bf 2009, fall 2011, spring 2012} \\
Conservation Biology Program, Fisheries and Wildlife Department, four semesters

{\bf Student Representative}, MN Chapter of Am. Fisheries Society \hfill {\bf Aug. 2009 - Aug. 2012} \\
Held student representative position for the Minnesota Chapter of the Am. Fisheries Society

{\bf Student Club Officer}, University of Minnesota \hfill {\bf Sep. 2010 - May 2012} \\
Served as vice-president of University of Minnesota Fisheries and Wildlife Club

{\bf Lab Volunteer}, Florida Museum of Natural History Shark Lab \hfill {\bf summer 2005} \\
Assisted marine biologists with preparation and maintenance of elasmobranch specimens

\sectitle{PUBLICATIONS}

\textbf{Beck, M.W.}, Hagy, J.D., Murrell, M.C. {\bf In review}. Improving estimates of ecosystem metabolism by reducing effects of tidal advection on dissolved oxygen time series. \textit{Limnology \& Oceanography: Methods}.

\textbf{Beck, M.W.}, Hagy, J.D. {\bf In press}. Adaptation of a weighted regression approach to evaluate water quality trends in an estuary. \textit{Environmental Modeling and Assessment}.

\textbf{Beck, M.W.}, Tomcko, C.M., Valley, R.D., Staples, D.F. 2014. Analysis of macrophyte indicator variation as a function of sampling, temporal, and stressor effects. \textit{Ecological Indicators}. 46:323-335.

\textbf{Beck, M.W.}, Wilson, B.N., Vondracek, B., Hatch, L.K. 2014. Application of neural networks to quantify utility of indices of biotic integrity for biological monitoring. \textit{Ecological Indicators}. 45:195-208.

Vondracek, B., Koch, J.D., and \textbf{Beck, M.W.} 2014. A comparison of survey methods to evaluate macrophyte index of biotic integrity performance in Minnesota lakes. \textit{Ecological Indicators}. 36:178-185.

\textbf{Beck, M.W.}, Vondracek, B. Hatch, L.K. 2013. Between- and within-lake responses of macrophyte richness metrics to shoreline development. \textit{Lake and Reservoir Management}. 29(3):179-193.

\textbf{Beck, M.W.}, Vondracek, B., Hatch, L.K., and Vinje, J. 2013. Semi-automated analysis of high-resolution aerial images to quantify docks in glacial lakes. \textit{ISPRS Journal of Photogrammetry and Remote Sensing}. 81:60-69.

\textbf{Beck, M.W.}, Vondracek, B., and Hatch, L.K. 2013. Environmental clustering of lakes to evaluate performance of a macrophyte index of biotic integrity. \textit{Aquatic Botany}. 108:16-25.

\textbf{Beck, M.W.}, Claassen, A.H., and Hundt, P.J. 2012. Environmental and livelihood impacts of dams: Common lessons across development gradients that challenge sustainability. \textit{International Journal of River Basin Management}. 10(1):73-92.

\textbf{Beck, M.W.}, Hatch, L.K., Vondracek, B., and Valley, R.D. 2010. Development of a macrophyte-based index of biotic integrity for Minnesota lakes. \textit{Ecological Indicators}. 10:968-979.

\textbf{Beck, M.W.}, and Hatch, L.K. 2009. A review of research on the development of lake indices of biotic integrity. \textit{Environmental Reviews}. 17:21-44.

\vspace{\baselineskip} 
\centerline{\large{\textit{TECHNICAL REPORTS AND SOFTWARE}}}

\textbf{Beck, M.W.} 2014. NeuralNetTools: Visualization and Analysis Tools for Neural Networks. Version 1.0.0. \href{http://cran.r-project.org/web/packages/NeuralNetTools/index.html}{http://cran.r-project.org/web/packages/NeuralNetTools/index.html}

\textbf{Beck, M.W.} 2014. SWMPr: An R package for the National Estuarine Research Reserve System. Version 1.4.0. \href{https://github.com/fawda123/SWMPr}{https://github.com/fawda123/SWMPr}

\textbf{Beck, M.W.} 2013. Minnesota macrophytes: Linking aquatic plants, lake health, and human activities. Doctoral Dissertation. University of Minnesota. 214 pp.

\textbf{Beck, M.W.} 2009. Development of an ecological assessment method for Minnesota lakes. Masters Thesis. University of Minnesota. 164 pp.

\textbf{Beck, M.W.} 2006. Reproductive characteristics of the green treefrog \textit{Hyla cinerea}: effects of agricultural contaminants. Honors Thesis. University of Florida. 32 pp.

\vspace{\baselineskip} 
\centerline{\large{\textit{INVITED PRESENTATIONS}}}

{\bf Beck, M.W.} 2014. The search for truth in numbers: Quantitative approaches for evaluating trends in water quality data. \textit{Oral presentation}. Biology Department seminar (PCB 3930/4922/5924), University of West Florida, Pensacola, FL. 

{\bf Beck, M.W.}, Vondracek, B., Wilson, B., and Hatch, L.K. 2013. Understanding indicators of lake health and the utility of a plant-based index. \textit{Oral presentation}. Gulf Ecology Division seminar series, USEPA NHEERL ORD, Gulf Breeze, FL.

{\bf Beck, M.W.}, Vondracek, B., Hatch, L.K., and Wilson, B. 2013. Minnesota macrophytes: Linking aquatic plants, lake integrity, and human activities. \textit{Poster presentation}. Doctoral Research Showcase, University of Minnesota, Minneapolis, MN.

{\bf Beck, M.W.}, Vondracek, B., and Hatch, L.K. 2012. Between and within lake responses of macrophyte richness metrics to shoreline development. \textit{Oral presentation}. MN Department of Natural Resources Fall research meeting, Itasca State Park, MN.

{\bf Beck, M.W.}, Vondracek, B., Hatch, L.K. 2011. Biological assessment of aquatic macrophytes in Midwest glacial lakes. \textit{Oral presentation}. University of Minnesota Water Resources Graduate Program student symposium, Stillwater, MN.

{\bf Beck, M.W.} 2009. Perspectives on sustaining Minnesota's lakes: Biological monitoring, IBIs, and stressor identification. \textit{Oral presentation}. University of Minnesota Water Resources Graduate Program seminar, St. Paul, MN.

{\bf Beck, M.W.} 2009. The Minnesota macrophyte IBI for lake assessment. \textit{Oral presentation}. Habitat research meeting, Minnesota Department of Natural Resources, St. Paul, MN.

{\bf Beck, M.W.}, Hatch, L.K., Vondracek, B., Valley, R.D. 2009. Development of a macrophyte-based index of biotic integrity for Minnesota lakes. \textit{Oral presentation}. Aquatic plant assessment meeting, Minnesota Pollution Control Agency, St. Paul, MN.

{\bf Beck, M.W.}, Hatch, L.K., Vondracek, B., Valley, R.D. 2009. Development of a macrophyte-based index of biotic integrity for Minnesota lakes. \textit{Oral presentation}. Fisheries research meeting, Minnesota Department of Natural Resources, Cloquet, MN.

\vspace{\baselineskip} 
\centerline{\large{\textit{CONTRIBUTED PRESENTATIONS}}}

{\bf Beck, M.W.}, Hagy, J.D., Murrell, M.C. 2014. A novel approach for evaluation of water quality trends in Gulf Coast estuaries. \textit{Oral presentation}. Bays and Bayous Symposium, Mobile, AL.

Murrell, M.C., Hagy, J.D., Aukamp, J., {\bf Beck, M.W.}, Beddick, D., Craven, G., Duffy, A., Jarvis, B.M., Marcovich, M., Yates, D., Caffrey, J. 2014. Environmental drivers of ecosystem and plankton metabolism in Pensacola Bay, Florida. Bays and Bayous Symposium, Mobile, AL. 

Hagy, J.D., Jarvis, B.M., Murrell, M.C., {\bf Beck, M.W.}. 2014. Modeling remediation of aquatic life impacts of episodic and diel cycling hypoxia via nutrient loading rate reduction. Bays and Bayous Symposium, Mobile, AL. 

{\bf Beck, M.W.} 2014. ggplot2 redux. \textit{Oral presentation}. USEPA R User Group meeting, USEPA NHEERL ORD, Gulf Breeze, FL.

{\bf Beck, M.W.}, Tomcko, C.M., Valley, R.D., and Staples, D.F. 2014. Analysis of macrophyte indicator variation as a function of sampling, temporal, and stressor effects. \textit{Poster presentation}. Joint Aqautic Sciences Meeting, Portland, OR.

{\bf Beck, M.W.}, Hagy, J.D. 2014. Adaptation of a weighted regression approach to evaluate water quality trends in {T}ampa {B}ay, {F}lorida. \textit{Oral presentation}. National Water Quality Monitoring Council annual meeting, Cincinnati, OH.

{\bf Beck, M.W.}. 2014. The nuts and bolts of Sweave/Knitr for reproducible research. \textit{Oral presentation}. USEPA R User Group meeting, USEPA NHEERL ORD, Gulf Breeze, FL.

{\bf Beck, M.W.}, Vondracek, B., Hatch, L.K., and Wilson, B. 2013. Evaluating the utility of a plant-based index of lake condition using neural networks. \textit{Oral presentation}. Ecological Society of America annual meeting, Minneapolis, MN.

{\bf Beck, M.W.}, Vondracek, B., Wilson, B., and Hatch, L.K. 2013. Understanding indicators of lake health and the utility of a plant-based index. \textit{Oral presentation}. Conservation Biology Graduate Program seminar, University of Minnesota, St. Paul, MN.

Vondracek, B., Koch, J.D., and {\bf Beck, M.W.} 2013. A comparison of survey methods to evaluate macrophyte index of biotic integrity performance in Minnesota lakes. \textit{Oral presentation}. Minnesota chapter of the Am. Fisheries Society annual meeting, St. Cloud, MN.

{\bf Beck, M.W.}, Vondracek, B., Wilson, B., and Hatch, L.K. 2013. Understanding indicators of lake health and the utility of a plant-based index. \textit{Oral presentation}. Conservation Biology Graduate Program seminar, University of Minnesota, St. Paul, MN.

{\bf Beck, M.W.} 2012. Effects of shoreline development on aquatic macrophytes: Summary of analyses. \textit{Class lecture}. Ecological Modeling (FW 8200), University of Minnesota, St. Paul, MN.

{\bf Beck, M.W.}, Vondracek, B., Hatch, L.K., and Vinje, J. 2012. Image analysis techniques to evaluate effects of lakeshore development on aquatic habitat. \textit{Oral and poster presentation}. Am. Fisheries Society annual meeting, St. Paul, MN.

{\bf Beck, M.W.}, Vondracek, B., Hatch, L.K., and Vinje, J. 2012. Image analysis techniques to evaluate effects of lakeshore development on aquatic habitat. \textit{Oral presentation}. Water Resources Conference, St. Paul, MN.

{\bf Beck, M.W.}, Vondracek, B., Hatch, L.K., 2012. Identifying covariates of a lake assessment index to improve biological assessment. \textit{Oral presentation}. Minnesota chapter of Society for Conservation Biology annual meeting, St. Paul, MN.

{\bf Beck, M.W.}, Vondracek, B., Hatch, L.K. 2011. Image analysis techniques to evaluate effects of nearshore lake development on aquatic macrophytes. \textit{Oral presentation}. Midwest Fish and Wildlife Conference, Des Moines, IA.

{\bf Beck, M.W.} 2011. R is my friend: Or how I learned to stop worrying and love the machine. \textit{Oral presentation}. Conservation Biology Graduate Program breakfast seminar, University of Minnesota, St.Paul, MN.

{\bf Beck, M.W.}, Vondracek, B., Hatch, L.K. 2011. Image analysis techniques to evaluate effects of nearshore lake development on aquatic macrophytes. \textit{Poster presentation}. Water Resources Conference, St. Paul, MN.

{\bf Beck, M.W.}, Vondracek, B., Hatch, L.K. 2011. Image analysis techniques to evaluate effects of nearshore lake development on aquatic macrophytes. \textit{Oral presentation}. Am. Fisheries Society annual meeting, Seattle, WA.

{\bf Beck, M.W.}, Vondracek, B., Hatch, L.K. 2011. Image analysis techniques to evaluate effects of nearshore lake development on aquatic macrophytes.\textit{Oral presentation}. Minnesota chapter of the Am. Fisheries Society annual meeting, Sandstone, MN.

{\bf Beck, M.W.}, Hatch, L.K., Vondracek, B., Valley, R.D. 2010. Development of a macrophyte-based index of biotic integrity for Minnesota lakes. \textit{Poster presentation}. Midwest Fish and Wildlife Conference, Minneapolis, MN.

Koch, J.D., {\bf Beck, M.W.}, Carlin, J. 2010. A comparison of survey methods to evaluate macrophyte index of biotic integrity performance in Minnesota lakes. \textit{Poster presentation}. Midwest Fish and Wildlife Conference, Minneapolis, MN.

{\bf Beck, M.W.}, Hatch, L.K., Vondracek, B., Valley, R.D. 2010. Development of a macrophyte-based index of biotic integrity for Minnesota lakes. \textit{Oral presentation}. International Congress for Conservation Biology, Edmonton, Alberta.

{\bf Beck, M.W.}, Vondracek, B., Hatch, L.K. 2010. Assessing the health of Minnesota's lakes using indices of biotic integrity. \textit{Oral presentation}. Joint meeting of the Minnesota chapters of Am. Fisheries Society, Society for Conservation Biology, and The Wildlife Society, Nisswa, MN.

{\bf Beck, M.W.}, Hatch, L.K., Vondracek, B., Valley, R.D. 2009. Development of a macrophyte-based index of biotic integrity for Minnesota lakes. \textit{Oral presentation}. Water Resources Conference, St. Paul, MN.

{\bf Beck, M.W.}, Hatch, L.K., Vondracek, B., Valley, R.D. 2009. Development of a macrophyte-based index of biotic integrity for Minnesota lakes. \textit{Poster presentation}. Land Conservation and Clean Water Summit, Chaska, MN.

{\bf Beck, M.W.}, Hatch, L.K., Vondracek, B., Valley, R.D. 2009. Development of a macrophyte-based index of biotic integrity for Minnesota lakes. \textit{Oral presentation}. Symposium for best student paper, Am. Fisheries Society annual meeting, Nashville, TN.

{\bf Beck, M.W.}, Hatch, L.K., Vondracek, B., Valley, R.D. 2009. Development of a macrophyte-based index of biotic integrity for Minnesota lakes. \textit{Oral presentation}. Minnesota chapter of the Society for Conservation Biology annual meeting, St. Paul, MN.

{\bf Beck, M.W.}, Hatch, L.K., Vondracek, B., Valley, R.D. 2008. Development of a macrophyte-based index of biotic integrity for Minnesota lakes. \textit{Poster presentation}. Water Resources Conference, St. Paul, MN.

{\bf Beck, M.W.} 2008. The index of biotic integrity: A review. \textit{Class lecture}. Water quality and natural resources (ESPM 4061), University of Minnesota, St. Paul, MN.

{\bf Beck, M.W.}, Hatch, L.K., Vondracek, B., Valley, R.D. 2008. Development of a macrophyte-based index of biotic integrity for Minnesota lakes. \textit{Oral presentation}. Conservation Biology Graduate Program seminar, University of Minnesota, St. Paul, MN.

Hatch, L.K., {\bf Beck, M.W.}, Vondracek, B. 2007. Ecological assessment method development for Minnesota lakes. \textit{Oral presentation}. North Am. Lake Management Society annual meeting, Orlando, FL.

\sectitle{HONORS AND AWARDS}

LaBounty Best Paper Award nomination, N. Am. Lake Management Society \hfill {\bf Oct. 2014} \\
Interdisciplinary Doctoral Fellowship, University of Minnesota {\bf \$22500} \hfill {\bf Sep. 2012} \\
Best student paper award, MN chapter for Society of Conservation Biology \hfill {\bf Mar. 2012} \\
Carolyn Crosby Fellowship, University of Minnesota {\bf \$1000} \hfill {\bf Jan. 2012} \\
Travel award, Conservation Biology Graduate Program {\bf \$300} \hfill {\bf Sep. 2011} \\
Best student paper award, MN chapter of Am. Fisheries Society {\bf \$200} \hfill {\bf Feb. 2011} \\
Travel award, Minnesota chapter of Am. Fisheries Society {\bf \$65} \hfill {\bf Feb. 2011} \\
Fenske Memorial Award finalist, Midwest Fish and Wildlife Conference \hfill {\bf Dec. 2010} \\
Travel award, Strategic Env. Research and Development Program {\bf \$700} \hfill {\bf Jul. 2010} \\
Travel award, Conservation Biology Graduate Program {\bf \$300} \hfill {\bf Jul. 2010} \\
Travel award, Graduate and Professional Student Assembly {\bf \$200} \hfill {\bf Jul. 2010} \\
Travel award, Minnesota chapter of Am. Fisheries Society {\bf \$103} \hfill {\bf Mar. 2010} \\
Finalist in best student paper symposium, Am. Fisheries Society \hfill {\bf Sep. 2009} \\
Travel award, Conservation Biology Graduate Program {\bf \$400} \hfill {\bf Sep. 2009} \\
Travel award, Graduate and Professional Student Assembly {\bf \$165} \hfill {\bf Sep. 2009} \\
Conservation Biology Graduate Program summer fellowship {\bf \$3500} \hfill {\bf summer 2008} \\
Elected to Phi Beta Kappa Honor Society \hfill {\bf Dec. 2006} \\
Dean's list for academic excellence duration of undergraduate education \hfill {\bf 2004 - 2006} \\
Florida Bright Futures undergraduate scholarship {\bf \$5250} \hfill {\bf 2003 - 2006} 

\sectitle{PROFESSIONAL SOCIETY MEMBERSHIP}

Coastal and Estuarine Research Federation \hfill {\bf Sep. 2013 - Present} \\
Midwest Aquatic Plant Management Society \hfill {\bf Jan. 2012 - Present} \\
Minnesota Native Plant Society \hfill {\bf Mar. 2011 - Present} \\
Society for Conservation Biology \hfill {\bf Feb. 2008 - Present} \\
American Fisheries Society \hfill {\bf Feb. 2008 - Present}

\sectitle{SKILLS}

R statistical software with RStudio Integrated Development Environment \\
Python programming language for geo-spatial modeling \\
Version control with Git and Github \\
\LaTeX\ typesetting including Sweave/knitR integration with R \\
ArcGIS software\\
ERDAS Imagine 2010 image analysis software \\
Microsoft Office including Access \\
Aquatic plant and fish identification of Minnesota lakes \\
Standard survey techniques for aquatic plants and fishes of Minnesota lakes \\
Boat/trailer operation \\
CPR and first aid certification

\sectitle{ADDITIONAL EXPERIENCE}

Author of Github repositories - \url{http://github.com/fawda123/} \\
Blog contributor for R-bloggers - \url{http://beckmw.wordpress.com/} \\
Workshop participant, Master R Developer, San Francisco, CA \\
Webinar participant, The Grammar and Graphics of Data Science \\
Workshop participant, New Tools for Water Quality Data Access and Trend \\ \hspace{0.3in} Analysis. An Overview of the USGS R Packages: dataRetrieval and EGRET\\
Webinar participant, Revolution Analytics Introduction to R for Data Mining \\
Workshop participant, An Introduction to the Fundamentals of Linear \\
\hspace{0.3in}Quantile Regression in R \\
On-line course, Principles and Techniques of Electrofishing, US FWS \\
Workshop participant, Simple Tools for Lake and Watershed Trophic \\
\hspace{0.3in}Assessment (FLUX32, Bathtub, TASTR), St. Paul, MN \\
Workshop participant, Aquatic Plant Identification, Itasca State Park, MN 

\end{document}