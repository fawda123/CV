\documentclass[letterpaper,12pt]{article} 
\usepackage{times}
\usepackage[top=1in,bottom=1in,left=1in,right=1in]{geometry}
\usepackage{graphics}
\usepackage{booktabs}
\usepackage{fancyhdr}
\usepackage{xcolor}
\usepackage{enumitem}
\usepackage{url}
\usepackage{datetime}

\newcommand{\sectitle}[1]{\vspace{\baselineskip} \centerline{\large{\textit{#1}}}} 
\newdateformat{mydate}{\monthname[\THEMONTH] \THEYEAR}
\urlstyle{same}

\setlist{noitemsep,topsep=0pt,parsep=0pt,partopsep=0pt,leftmargin=*}
\setlength{\parindent}{0pt}
\setlength{\parskip}{\baselineskip}%
\pagestyle{fancy}
\fancyhf{}
\fancyfoot[R]{\textcolor{gray}{Marcus W. Beck, CV page \thepage}}
\fancyfoot[L]{\textcolor{gray}{\mydate\today}}
\renewcommand{\headrulewidth}{0pt}
\renewcommand{\rmdefault}{ptm}

\begin{document}

\raggedright

\LARGE
\centerline{{\bf Dr. Marcus William Beck}}
\normalsize
\textit{USEPA NHEERL Gulf Ecology Division \hfill Office: 850-934-2480 \\
1 Sabine Island Drive \hfill Cell: 352-871-6795 \\
Gulf Breeze, FL 32561 \hfill Email: beck.marcus@epa.gov}
\vspace{4pt}
\hrule
\vspace{2pt}
\hrule
\vspace{4pt}

\sectitle{INTERESTS}

Water quality modeling, data mining, geographic information systems and remote sensing, reproducible research, multivariate statistics and neural networks, land-water interactions, aquatic ecology, ecological indicators, water quality policy

\sectitle{EDUCATION}

{\bf Ph.D., Conservation Biology, Fisheries and Aquatic Biology track} \hfill Jun. 2013 \\
{\bf Statistics minor} \\
University of Minnesota, Twin Cities, MN \\
Advisors: Drs. Bruce Vondracek, Lorin Hatch; GPA 3.94 \\
Dissertation: Minnesota macrophytes: Linking aquatic plants, lake health, and human activities\\
Coursework in applied regression analysis, aquatic entomology, ecological modeling, fish habitats and restoration, and theory of statistics

{\bf M.Sc., Conservation Biology, Fisheries and Aquatic Biology track} \hfill Jun. 2009 \\
University of Minnesota, Twin Cities, MN \\
Advisor: Dr. Lorin Hatch; GPA 4.00 \\
Thesis: Development of an ecological assessment method for Minnesota lakes \\
Coursework in conservation biology, statistics, geographic information systems, limnology, ichthyology, and fisheries population analysis, field course in wetland and lakeshore plant identification

{\bf B.Sc., Zoology} \textit{summa cum laude} \hfill Dec. 2006 \\
University of Florida, Gainesville, FL \\
Advisor: Dr. Krista McCoy; GPA 3.81 \\
Honors thesis: Reproductive characteristics of the green tree frog (Hyla cinerea): Effects of agricultural contaminants \\
Coursework in anatomy, biology, ecology, genetics, and reproductive endocrinology

{\bf A.A., Zoology} \hfill Apr. 2004 \\
Santa Fe Community College, Gainesville, FL \\
GPA 3.87

\sectitle{EXPERIENCE}

{\bf Post-Doctorate Research Fellow}, USEPA \hfill Jul. 2013 - Present\\
\textit{Supervisor}: Dr. Jim Hagy \\
Held post-doctorate research fellowship position in Oak Ridge Institute for Science and Education at USEPA Gulf Ecology Division, Ecosystem Dynamics and Effects Branch; Developed and applied watershed models, hydrodynamic water quality models, and population models for estuarine systems to investigate impacts of anthropogenic changes in the watershed on ecosystem condition and function 

{\bf Research Fellow}, University of Minnesota \hfill Sep. 2012 - Jun. 2013\\
\textit{Supervisor}: Dr. Bruce Wilson \\
Analyzed utility of biological monitoring methods with statewide datasets and multivariate analyses (neural networks), including identification of land use stressors that influence lake condition; Prepared progress reports and public presentations to communicate results and provide management recommendations

{\bf Short Course Co-Instructor}, University of Minnesota \hfill May 2013\\
\textit{Supervisor}: Dr. Susan Galatowitsch \\
Co-instructed two short courses on basic data analysis that were part of a larger workshop focused on ecosystem restoration; Provided instruction using interactive lectures and take-home material as an introduction to software R

{\bf Research Assistant}, University of Minnesota \hfill Sep. 2007 - Sep. 2012\\
\textit{Supervisor}: Dr. Bruce Vondracek \\
Developed an index of biotic integrity using aquatic macrophytes to determine the ecological health of Minnesota lakes; Developed remote sensing and image analysis techniques to quantify land use stressors from high-resolution aerial photos; Assisted researchers at the Minnesota Department of Natural Resources to develop appropriate methods for long-term lake monitoring program

{\bf Primary Instructor}, University of Minnesota \hfill Jan. 2012 - May 2012 \\
\textit{Supervisor}: Dr. Francesca Cuthbert \\
Instructed FW 5604 (Fisheries Ecology and Management, 4 cr.) during Spring 2012 semester; Administered 44 lectures, nine discussions, five homework assignments, and three exams for 11 students to develop understanding of fundamental topics in fisheries ecology and management

{\bf Undergraduate Advisor}, University of Minnesota \hfill Mar. 2009 - May 2011 \\
\textit{Supervisor}: Dr. Bruce Vondracek, Dr. Lorin K. Hatch \\
Advised undergraduate honors project that examined survey designs for biological assessment of aquatic plants; Advised undergraduate intern with Minnesota Department of Natural Resources that examined use of fish indices to assess lake condition 

{\bf Intern}, Minnesota Department of Natural Resources \hfill Jun. 2009 - Aug. 2009 \\
\textit{Supervisor}: Timothy Cross \\
Assisted in collection, preparation, and analysis of data in support of lake monitoring research; Collaborated with state biologists to devise scientific approaches that evaluate effects of climate change on lake condition; Presented research at agency meetings

{\bf Teaching Assistant}, University of Minnesota \hfill Sep. 2008 - Dec. 2008 \\
\textit{Supervisor}: Dr. Joseph Magner \\
Held teaching assistant position for ESPM 4061/5061 (Water Quality and Natural Resources, 3cr.); Administered guest-lecture, exams, and assignments for 25 students

{\bf Creel Clerk}, Florida Fish and Wildlife Commission \hfill Jan. 2007 - May 2007 \\
\textit{Supervisor}: Eric Nagid \\
Conducted boat creel interviews of anglers for catch and effort of black crappie (\textit{Pomoxis nigromaculatus}); Conducted laboratory work on black crappie including morphological measurements, otolith extraction, and year-class aging to support long-term monitoring program

{\bf Undergraduate Researcher}, University of Florida \hfill Jan. 2006 - Dec. 2006 \\
\textit{Supervisors}: Dr. Krista McCoy, Dr. Colette St. Mary \\
Investigated effects of pesticides on green tree frogs (\textit{Hyla cinerea}); Performed slide histology and tissue staining

{\bf Biological Technician}, US Geological Survey \hfill summer 2006 \\
\textit{Supervisor}: Noel Burkhead \\
Assisted with investigation of red shiner (\textit{Cyprinella lutrensis}) invasion; Collected mortality data on lab-reared fishes; Conducted stream sampling

\sectitle{SERVICE}

{\bf Peer Reviewer}, various journals \hfill Mar. 2010 - Present \\
Reviewer for Aquatic Botany, Development in Practice, Ecological Indicators, Ecological Research, Environmental Management, Estuaries and Coasts, Hydrobiologia, International Journal of Tropical Biology and Conservation, International Journal of River Basin Management, Journal of the North American Benthological Society, Lake and Reservoir Management, Marine Ecology, North American Journal of Fisheries Management, Polar Biology, Remote Sensing, Science of the Total Environment

{\bf Graduate student representative}, University of Minnesota \hfill Jan. 2013 - May 2013 \\
Served as graduate student representative and voting member for faculty meetings in the Fisheries and Wildlife Department

{\bf Planning Committee Co-Chair}, American Fisheries Society \hfill Jun. 2010 - Aug. 2012 \\
Served as student activities committee co-chair for 2012 annual meeting of the American Fisheries Society; Organized student social, career fair, student colloquium for 350 students

{\bf Fundraising Coordinator}, University of Minnesota \hfill spring 2011, spring 2012 \\
Committee chair for Conservation Biology Graduate Program fundraiser in spring 2012, active fundraiser in spring 2011; Raised approximately \$6000 in funds for student travel to professional conferences and workshops; Logistical management of bike relay across Minnesota

{\bf Seminar Coordinator}, University of Minnesota \hfill fall 2011, spring 2012 \\
Organized Fisheries and Wildlife Department seminar for two semesters; Coordinated activities between University and external speakers

{\bf Student Representative}, MN Chapter of American Fisheries Society \hfill Aug. 2009 - Aug. 2012 \\
Held student representative position for the Minnesota Chapter of the American Fisheries Society; Recruited student members; Disseminated award/scholarship information; Attended chapter Executive Committee meetings

{\bf Student Club Officer}, University of Minnesota \hfill Sep. 2010 - May 2012 \\
Served as vice-president of University of Minnesota Fisheries and Wildlife Club; Recruited student members; Organized monthly meetings including outside speakers

{\bf Seminar Coordinator}, University of Minnesota \hfill spring 2009, fall 2009  \\
Organized Conservation Biology Graduate Program seminar for two semesters; Coordinated activities between University and external speakers

{\bf Lab Volunteer}, Florida Museum of Natural History Shark Lab \hfill summer 2005 \\
Assisted marine biologists with preparation and maintenance of elasmobranch specimens

\sectitle{PUBLICATIONS}

\textbf{Beck, M.W.}, Tomcko, C.M., Valley, R.D., Staples, D.F. 2014. Analysis of macrophyte indicator variation as a function of sampling, temporal, and stressor effects. \textit{Ecological Indicators}. 46:323-335.

\textbf{Beck, M.W.}, Wilson, B.N., Vondracek, B., Hatch, L.K. 2014. Application of neural networks to quantify utility of indices of biotic integrity for biological monitoring. \textit{Ecological Indicators}. 45:195-208.

Vondracek, B., Koch, J.D., and \textbf{Beck, M.W.} 2014. A comparison of survey methods to evaluate macrophyte index of biotic integrity performance in Minnesota lakes. \textit{Ecological Indicators}. 36:178-185.

\textbf{Beck, M.W.}, Vondracek, B. Hatch, L.K. 2013. Between- and within-lake responses of macrophyte richness metrics to shoreline development. \textit{Lake and Reservoir Management}. 29(3):179-193.

\textbf{Beck, M.W.}, Vondracek, B., Hatch, L.K., and Vinje, J. 2013. Semi-automated analysis of high-resolution aerial images to quantify docks in glacial lakes. \textit{ISPRS Journal of Photogrammetry and Remote Sensing}. 81:60-69.

\textbf{Beck, M.W.}, Vondracek, B., and Hatch, L.K. 2013. Environmental clustering of lakes to evaluate performance of a macrophyte index of biotic integrity. \textit{Aquatic Botany}. 108:16-25.

\textbf{Beck, M.W.}, Claassen, A.H., and Hundt, P.J. 2012. Environmental and livelihood impacts of dams: Common lessons across development gradients that challenge sustainability. \textit{International Journal of River Basin Management}. 10(1):73-92.

\textbf{Beck, M.W.}, Hatch, L.K., Vondracek, B., and Valley, R.D. 2010. Development of a macrophyte-based index of biotic integrity for Minnesota lakes. \textit{Ecological Indicators}. 10:968-979.

\textbf{Beck, M.W.}, and Hatch, L.K. 2009. A review of research on the development of lake indices of biotic integrity. \textit{Environmental Reviews}. 17:21-44.

\vspace{\baselineskip} 
\centerline{\large{\textit{PRESENTATIONS}}}

{\bf Beck, M.W.}. 2014. ggplot2 redux. \textit{Oral presentation}. USEPA R User Group meeting, USEPA NHEERL ORD, Gulf Breeze, FL.

{\bf Beck, M.W.}, Tomcko, C.M., Valley, R.D., and Staples, D.F. 2014. Analysis of macrophyte indicator variation as a function of sampling, temporal, and stressor effects. \textit{Poster presentation}. Joint Aqautic Sciences Meeting, Portland, OR.

{\bf Beck, M.W.}. 2014. Adaptation of a weighted regression approach to evaluate water quality trends in {T}ampa {B}ay, {F}lorida. \textit{Oral presentation}. National Water Quality Monitoring Council annual meeting, Cincinnati, OH.

{\bf Beck, M.W.}. 2014. The nuts and bolts of Sweave/Knitr for reproducible research. \textit{Oral presentation}. USEPA R User Group meeting, USEPA NHEERL ORD, Gulf Breeze, FL.

{\bf Beck, M.W.}, Vondracek, B., Hatch, L.K., and Wilson, B. 2013. Evaluating the utility of a plant-based index of lake condition using neural networks. \textit{Oral presentation}. Ecological Society of America annual meeting, Minneapolis, MN.

{\bf Beck, M.W.}, Vondracek, B., Wilson, B., and Hatch, L.K. 2013. Understanding indicators of lake health and the utility of a plant-based index. \textit{Oral presentation}. Gulf Ecology Division seminar series, USEPA NHEERL ORD, Gulf Breeze, FL.

{\bf Beck, M.W.}, Vondracek, B., Hatch, L.K., and Wilson, B. 2013. Minnesota macrophytes: Linking aquatic plants, lake integrity, and human activities. \textit{Invited/poster presentation}. Doctoral Research Showcase, University of Minnesota, Minneapolis, MN.

{\bf Beck, M.W.}, Vondracek, B., Wilson, B., and Hatch, L.K. 2013. Understanding indicators of lake health and the utility of a plant-based index. \textit{Oral presentation}. Conservation Biology Graduate Program seminar, University of Minnesota, St. Paul, MN.

Vondracek, B., Koch, J.D., and {\bf Beck, M.W.} 2013. A comparison of survey methods to evaluate macrophyte index of biotic integrity performance in Minnesota lakes. \textit{Oral presentation}. Minnesota chapter of the American Fisheries Society annual meeting, St. Cloud, MN.

{\bf Beck, M.W.}, Vondracek, B., Wilson, B., and Hatch, L.K. 2013. Understanding indicators of lake health and the utility of a plant-based index. \textit{Oral presentation}. Conservation Biology Graduate Program seminar, University of Minnesota, St. Paul, MN.

{\bf Beck, M.W.} 2012. Effects of shoreline development on aquatic macrophytes: Summary of analyses. \textit{Class lecture}. Ecological Modeling (FW 8200), University of Minnesota, St. Paul, MN.

{\bf Beck, M.W.}, Vondracek, B., and Hatch, L.K. 2012. Between and within lake responses of macrophyte richness metrics to shoreline development. \textit{Oral presentation}. MN Department of Natural Resources Fall research meeting, Itasca State Park, MN.

{\bf Beck, M.W.}, Vondracek, B., Hatch, L.K., and Vinje, J. 2012. Image analysis techniques to evaluate effects of lakeshore development on aquatic habitat. \textit{Oral and poster presentation}. American Fisheries Society annual meeting, St. Paul, MN.

{\bf Beck, M.W.}, Vondracek, B., Hatch, L.K., and Vinje, J. 2012. Image analysis techniques to evaluate effects of lakeshore development on aquatic habitat. \textit{Oral presentation}. Water Resources Conference, St. Paul, MN.

{\bf Beck, M.W.}, Vondracek, B., Hatch, L.K., 2012. Identifying covariates of a lake assessment index to improve biological assessment. \textit{Oral presentation}. Minnesota chapter of Society for Conservation Biology annual meeting, St. Paul, MN.

{\bf Beck, M.W.}, Vondracek, B., Hatch, L.K. 2011. Image analysis techniques to evaluate effects of nearshore lake development on aquatic macrophytes. \textit{Oral presentation}. Midwest Fish and Wildlife Conference, Des Moines, IA.

{\bf Beck, M.W.} 2011. R is my friend: Or how I learned to stop worrying and love the machine. \textit{Oral presentation}. Conservation Biology Graduate Program breakfast seminar, University of Minnesota, St.Paul, MN.

{\bf Beck, M.W.}, Vondracek, B., Hatch, L.K. 2011. Image analysis techniques to evaluate effects of nearshore lake development on aquatic macrophytes. \textit{Poster presentation}. Water Resources Conference, St. Paul, MN.

{\bf Beck, M.W.}, Vondracek, B., Hatch, L.K. 2011. Biological assessment of aquatic macrophytes in Midwest glacial lakes. \textit{Invited/oral presentation}. University of Minnesota Water Resources Graduate Program student symposium, Stillwater, MN.

{\bf Beck, M.W.}, Vondracek, B., Hatch, L.K. 2011. Image analysis techniques to evaluate effects of nearshore lake development on aquatic macrophytes. \textit{Oral presentation}. American Fisheries Society annual meeting, Seattle, WA.

{\bf Beck, M.W.}, Vondracek, B., Hatch, L.K. 2011. Image analysis techniques to evaluate effects of nearshore lake development on aquatic macrophytes.\textit{Oral presentation}. Minnesota chapter of the American Fisheries Society annual meeting, Sandstone, MN.

{\bf Beck, M.W.}, Hatch, L.K., Vondracek, B., Valley, R.D. 2010. Development of a macrophyte-based index of biotic integrity for Minnesota lakes. \textit{Poster presentation}. Midwest Fish and Wildlife Conference, Minneapolis, MN.

{\bf Koch, J.D.}, Beck, M.W., Carlin, J. 2010. A comparison of survey methods to evaluate macrophyte index of biotic integrity performance in Minnesota lakes. \textit{Poster presentation}. Midwest Fish and Wildlife Conference, Minneapolis, MN.

{\bf Beck, M.W.}, Hatch, L.K., Vondracek, B., Valley, R.D. 2010. Development of a macrophyte-based index of biotic integrity for Minnesota lakes. \textit{Oral presentation}. International Congress for Conservation Biology, Edmonton, Alberta.

{\bf Beck, M.W.} 2010. The misshapen ball of clay: Adventures in a PhD program. \textit{Oral presentation}. Habitat research meeting, Minnesota Department of Natural Resources, Brainerd, MN.

{\bf Beck, M.W.}, Vondracek, B., Hatch, L.K. 2010. Assessing the health of Minnesota's lakes using indices of biotic integrity. \textit{Oral presentation}. Joint meeting of the Minnesota chapters of American Fisheries Society, Society for Conservation Biology, and The Wildlife Society, Nisswa, MN.

{\bf Beck, M.W.}, Hatch, L.K., Vondracek, B., Valley, R.D. 2009. Development of a macrophyte-based index of biotic integrity for Minnesota lakes. \textit{Oral presentation}. Water Resources Conference, St. Paul, MN.

{\bf Beck, M.W.} 2009. Perspectives on sustaining Minnesota's lakes: Biological monitoring, IBIs, and stressor identification. \textit{Invited/oral presentation}. University of Minnesota Water Resources Graduate Program seminar, St. Paul, MN.

{\bf Beck, M.W.}, Hatch, L.K., Vondracek, B., Valley, R.D. 2009. Development of a macrophyte-based index of biotic integrity for Minnesota lakes. \textit{Poster presentation}. Land Conservation and Clean Water Summit, Chaska, MN.

{\bf Beck, M.W.}, Hatch, L.K., Vondracek, B., Valley, R.D. 2009. Development of a macrophyte-based index of biotic integrity for Minnesota lakes. \textit{Oral presentation}. Finalist in symposium for best student paper, American Fisheries Society annual meeting, Nashville, TN.

{\bf Beck, M.W.} 2009. The Minnesota macrophyte IBI for lake assessment. \textit{Oral presentation}. Habitat research meeting, Minnesota Department of Natural Resources, St. Paul, MN.

{\bf Beck, M.W.}, Hatch, L.K., Vondracek, B., Valley, R.D. 2009. Development of a macrophyte-based index of biotic integrity for Minnesota lakes. \textit{Oral presentation}. Aquatic plant assessment meeting, Minnesota Pollution Control Agency, St. Paul, MN.

{\bf Beck, M.W.}, Hatch, L.K., Vondracek, B., Valley, R.D. 2009. Development of a macrophyte-based index of biotic integrity for Minnesota lakes. \textit{Oral presentation}. Minnesota chapter of the Society for Conservation Biology annual meeting, St. Paul, MN.

{\bf Beck, M.W.}, Hatch, L.K., Vondracek, B., Valley, R.D. 2009. Development of a macrophyte-based index of biotic integrity for Minnesota lakes. \textit{Oral presentation}. Fisheries research meeting, Minnesota Department of Natural Resources, Cloquet, MN.

{\bf Beck, M.W.} 2008. Fish reproduction in an evolutionary context. \textit{Class lecture}. Biology of fishes (FW 5136), University of Minnesota, St. Paul, MN.

{\bf Beck, M.W.}, Hatch, L.K., Vondracek, B., Valley, R.D. 2008. Development of a macrophyte-based index of biotic integrity for Minnesota lakes. \textit{Poster presentation}. Water Resources Conference, St. Paul, MN.

{\bf Beck, M.W.} 2008. The index of biotic integrity: A review. \textit{Class lecture}. Water quality and natural resources (ESPM 4061), University of Minnesota, St. Paul, MN.

{\bf Beck, M.W.}, Hatch, L.K., Vondracek, B., Valley, R.D. 2008. Development of a macrophyte-based index of biotic integrity for Minnesota lakes. \textit{Oral presentation}. Conservation Biology Graduate Program seminar, University of Minnesota, St. Paul, MN.

{\bf Beck, M.W.} 2008. The littoral and alternate stable states. \textit{Class lecture}. Limnology (EEB 5601), University of Minnesota, St. Paul, MN.

{\bf Hatch, L.K.}, Beck, M.W., Vondracek, B. 2007. Ecological assessment method development for Minnesota lakes. \textit{Oral presentation}. North American Lake Management Society annual meeting, Orlando, FL.

\sectitle{HONORS AND AWARDS}

Interdisciplinary Doctoral Fellowship, University of Minnesota {\bf \$22500} \hfill Sep. 2012 \\
Best student paper award, Minnesota chapter for Society of Conservation Biology \hfill Mar. 2012 \\
Carolyn Crosby Fellowship, University of Minnesota {\bf \$1000} \hfill Jan. 2012 \\
Travel award, Conservation Biology Graduate Program {\bf \$300} \hfill Sep. 2011 \\
Best student paper award, Minnesota chapter of American Fisheries Society {\bf \$200} \hfill Feb. 2011 \\
Travel award, Minnesota chapter of American Fisheries Society {\bf \$65} \hfill Feb. 2011 \\
Fenske Memorial Award finalist, Midwest Fish and Wildlife Conference \hfill Dec. 2010 \\
Travel award, Strategic Environmental Research and Development Program {\bf \$700} \hfill Jul. 2010 \\
Travel award, Conservation Biology Graduate Program {\bf \$300} \hfill Jul. 2010 \\
Travel award, Graduate and Professional Student Assembly {\bf \$200} \hfill Jul. 2010 \\
Travel award, Minnesota chapter of American Fisheries Society {\bf \$103} \hfill Mar. 2010 \\
Travel award, Conservation Biology Graduate Program {\bf \$400} \hfill Sep. 2009 \\
Travel award, Graduate and Professional Student Assembly {\bf \$165} \hfill Sep. 2009 \\
Conservation Biology Graduate Program summer fellowship {\bf \$3500} \hfill summer 2008 \\
Elected to Phi Beta Kappa Honor Society \hfill Dec. 2006 \\
Dean's list for academic excellence duration of undergraduate education \hfill 2004 - 2006 \\
Florida Bright Futures undergraduate scholarship {\bf \$5250} \hfill 2003 - 2006 

\sectitle{PROFESSIONAL SOCIETY MEMBERSHIP}

Coastal and Estuarine Research Federation \hfill Sep. 2013 - Present \\
Midwest Aquatic Plant Management Society \hfill Jan. 2012 - Present \\
Minnesota Native Plant Society \hfill Mar. 2011 - Present \\
Society for Conservation Biology \hfill Feb. 2008 - Present \\
Society for Conservation Biology, Minnesota chapter \hfill Apr. 2009 - Present \\
American Fisheries Society \hfill Feb. 2008 - Present \\
American Fisheries Society, Minnesota chapter \hfill Sep. 2009 - Present

\sectitle{SKILLS}

R statistical software with Tinn-R text editor and RStudio \\
Python programming language for geo-spatial modeling \\
Version control with Git and Github \\
Web scraping and multivariate analyses with Python and R \\
\LaTeX\ typesetting including Sweave/knitR integration with R \\
ArcGIS software including geo-processing and model building\\
ERDAS Imagine 2010 image analysis software \\
Microsoft Office including Access \\
Aquatic plant and fish identification of Minnesota lakes \\
Standard survey techniques for aquatic plants and fishes of Minnesota lakes \\
Boat/trailer operation \\
CPR and first aid certification

\sectitle{ADDITIONAL EXPERIENCE}

Blog contributor for R-bloggers - \url{http://beckmw.wordpress.com/} \hfill Dec. 2012 - Present \\
Webiner participant, The Grammar and Graphics of Data Science \hfill Jul. 2014\\
Workshop participant, New Tools for Water Quality Data Access and Trend \hfill May 2014\\ \hspace{0.3in} Analysis. An Overview of the USGS R Packages: dataRetrieval and EGRET\\
Webinar participant, Revolution Analytics introduction to R for data mining \hfill Feb. 2013 \\
Workshop participant, An Introduction to the Fundamentals of Linear \hfill Aug. 2010 \\
\hspace{0.3in}Quantile Regression in R \\
On-line course, Principles and Techniques of Electrofishing, US FWS \hfill Jul. 2010 \\
Workshop participant, Simple Tools for Lake and Watershed Trophic \hfill Jul. 2009 \\
\hspace{0.3in}Assessment (FLUX32, Bathtub, TASTR), St. Paul, MN \\
Workshop participant, Aquatic Plant Identification, Itasca State Park, MN \hfill Jun. 2009

\end{document}